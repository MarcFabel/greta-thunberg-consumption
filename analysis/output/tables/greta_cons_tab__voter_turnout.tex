\begin{table}[t]\centering
	\begin{threeparttable}
		\caption{Strike participation and voter turnout}\label{tab_greta_cons:associations_part_turnout}
		{\def\sym#1{\ifmmode^{#1}\else\(^{#1}\)\fi} 
			\begin{tabular*}{.68\linewidth}{@{\extracolsep{\fill}}l*{3}{c}}
				\toprule
				&\multicolumn{1}{c}{(1)}&\multicolumn{1}{c}{(2)}&\multicolumn{1}{c}{(3)}\\
				& Turnout & $\Delta$ Turnout & $\Delta$ Turnout \\
				& 2019		 & 2019-2015& 2019-2015	\\
				\midrule
				Participation Index [std.]	&      0.0140		  &      0.0787\sym{***}	&	0.1055\sym{***}			\\
											&    (0.0325)         &    (0.0181)     		&	(0.0219)    				\\
				\\	
				Observations        		&      12,189         &      12,189         	&	455		\\
				Adjusted $R^2$         		&       0.270         &       0.499         	&	0.785	\\
				State FE					& \checkmark 		  & \checkmark 				& \checkmark \\
				Election FE					& \checkmark 		  & \checkmark       		& \checkmark \\
				\bottomrule
		\end{tabular*}}
		\begin{tablenotes} 
			\item \scriptsize \emph{Notes:} The specifications use election results from the 2019 elections for the EU and the federal states of Brandenburg, Saxony, and Thuringia. The dependent variable is defined as the voter turnout, i.e. the number of votes relative to total votes cast. See Table \ref{tab_greta_cons:associations_part_greens} for additional details. Standard errors are clustered at the county level (number of clusters $=401$) in columns 1 and 2. Robust standard errors are reported in column 3. \newline Significance levels: * p < 0.10, ** p < 0.05, *** p < 0.01.
		\end{tablenotes} 
	\end{threeparttable}
\end{table}