\begin{figure}[t]\centering
	\caption{Strike participation for selected strikes}
	\label{fig_greta_cons:strike_participation_hh_ber}
	% Hamburg
	\begin{subfigure}[h]{0.45\linewidth}\centering
		\includegraphics[width=\linewidth]{descriptive/greta_cons_strike_participation_hh_ols.png}
	\end{subfigure}
	%Berlin
	\begin{subfigure}[h]{0.45\linewidth}\centering
		\includegraphics[width=\linewidth]{descriptive/greta_cons_strike_participation_ber_ols.png}
	\end{subfigure}
	%	%Aachen
	%	\begin{subfigure}[h]{0.4\linewidth}\centering
	%		\includegraphics[width=\linewidth]{descriptive/greta_cons_strike_participation_aa_ols.png}
	%	\end{subfigure}
	\begin{minipage}{0.9\linewidth}
		\scriptsize{\emph{Notes:} The maps show residualized movements (in thousand) for two exemplary strikes. The two strikes were both visited by Greta Thunberg and attracted large populations. A darker shade of green indicates that more people were coming to the climate strike conditional on the controls discussed in the text. The color scale classification is obtained by using the Fisher-Jenks natural breaks algorithm. The red dots mark the strikes' location, gray areas indicate missing data (censored), bold gray lines show state boundaries, and thin gray lines represent the regions defined by \textit{Teralytics}.}
	\end{minipage}
\end{figure}