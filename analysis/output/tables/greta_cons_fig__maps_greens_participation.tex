\begin{figure}[H]\centering
	% greens
	\begin{subfigure}[h]{0.49\linewidth}\centering
		\includegraphics[width=\linewidth]{descriptive/greta_cons_the_greens_eu_election_2019_ags8_100.png}
	\end{subfigure}
	% fd_greens
	\begin{subfigure}[h]{0.49\linewidth}\centering
		\includegraphics[width=\linewidth]{descriptive/greta_cons_fd_the_greens_eu_election_2019_ags8_100.png}
	\end{subfigure}
	\begin{subfigure}[h]{0.49\linewidth}\centering
		\includegraphics[width=\linewidth]{descriptive/greta_cons_particiaption_index_eu_election_2019_ags8_100.png}
	\end{subfigure}
	\begin{subfigure}[h]{0.49\linewidth}\centering
		\includegraphics[width=\linewidth]{descriptive/greta_cons_corr_particiaption_greens_eu_election_2019_ags8_200.png}
	\end{subfigure}
	\begin{minipage}{\linewidth}
		\caption{Spatial correlation of the vote share of the Greens and strike participation}
		\label{fig_greta_cons:spatial_correlation_greens_index}
		\scriptsize{\emph{Notes:} The maps show, at the municipality level, (A) the vote share for the Greens in the 2019 election, (B) the first-difference of the vote share for the greens (2019-2015), (C) the strike participation index at the time of the EU election (standardized), and (D) the bivariate distribution of the first-differenced vote share for the greens and the participation index. The color scales in panels A-C correspond to quintiles. To generate the bivariate color scale in panel D, we blend the two univariate scales (in terciles, $\Delta$ Greens in red and participation index in blue) into one. Bold lines indicate state boundaries, thin lines represent municipality borders.}
	\end{minipage}
\end{figure}\clearpage