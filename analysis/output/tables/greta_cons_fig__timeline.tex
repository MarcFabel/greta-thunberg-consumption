\newgeometry{left=2.8cm,right=2.8cm,top=3cm,bottom=3cm} 
\begin{landscape}
	\vspace*{\fill}
	\begin{figure}[H]\centering
		\begin{subfigure}[h]{0.85\linewidth}\centering\caption{Timeline of events}\centering
			\includegraphics[width=\linewidth]{descriptive/greta_cons_background_timeline_elections_events.pdf}
		\end{subfigure}
		
		\par\smallskip % force a bit of vertical whitespace
		\begin{subfigure}[h]{0.29\linewidth}\centering\caption{Appearance in print media}
			\includegraphics[width=\linewidth]{descriptive/greta_cons_fff_print_media_articles_2019.pdf}
		\end{subfigure}
		\begin{subfigure}[h]{0.29\linewidth}\centering\caption{Twitter feed of Greta Thunberg}
			\includegraphics[width=\linewidth]{descriptive/greta_cons_twitter_gt_favorites_retweets_2019.pdf}
		\end{subfigure}
		\begin{subfigure}[h]{0.29\linewidth}\centering\caption{Environmental awareness}
			\includegraphics[width=\linewidth]{descriptive/greta_cons_gesis_env_issue_combined}
		\end{subfigure}

		\begin{minipage}{\linewidth}
			\caption{Important events surrounding FFF, perception in (social) media, and environmental awareness in 2019}\label{fig_greta_cons:timeline_media_awareness}
			\scriptsize{\emph{Notes:} Panel a presents a timeline of relevant events over the year 2019. Above the timeline, the scheme lists elections that fall in the sample period (election for the European Parliament, Brandenburg, Saxony, and Thuringia). Below the timeline, there is a series of important events, either far-reaching strikes or critical milestones for Greta Thunberg. Panel b shows the daily share of print articles covering FFF (per thousand articles). An article is defined to report about FFF if it contains at least the keyword `Fridays for Future' (in various notations) or `climate strike'. The blue line stems from a moving average with a window of three days. Panel c plots the weekly number of favorites and retweets of Greta Thunberg's tweets (in thousand). Panel d presents the share of respondents who state that the environment is the most important issue in the representative election survey \cite{politbarometer2019}. Since 1977, the survey informs about the political mood, projections, and attitudes towards current topics.\newline \emph{Source:} Own representation with data from Genios, Twitter, and Forschungsgruppe Wahlen.}
		\end{minipage}
	\end{figure}
	\vspace*{\fill}\clearpage
\end{landscape}
\restoregeometry