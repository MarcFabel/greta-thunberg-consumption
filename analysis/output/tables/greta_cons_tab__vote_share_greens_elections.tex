\begin{table}[ht]\centering
	\begin{threeparttable}
		\caption{Vote share for the Greens across elections in 2019}
		\label{tab_greta_cons:data_greens_vote_share}
		\begin{tabular*}{.7\linewidth}{@{\extracolsep{\fill}}l*{5}{c}}
			\toprule
			%			&\multicolumn{1}{c}{(1)}&\multicolumn{1}{c}{(2)}&\multicolumn{1}{c}{(3)}&\multicolumn{1}{c}{(4)}&\multicolumn{1}{c}{(5)}\\
			
			&\multicolumn{2}{c}{Greens}&\multicolumn{2}{c}{$\Delta$ Greens}\\
			&\multicolumn{2}{c}{(2019)}&\multicolumn{2}{c}{(2019-2015)}\\
			\cmidrule(lr){2-3}\cmidrule(lr){4-5}
			Election		&	Mean		&	Sd		&	Mean	& Sd	&	Obs.	\\
			\midrule\\
			
			EU				&	20.762		&	7.537	&	9.731	& 3.859	&	10,719	\\
			Brandenburg		&	10.167		&	5.120	&	4.308	& 2.189	&	413		\\
			Saxony			&	4.937		&	1.906	&	1.305	& 0.905	&	414		\\
			Thuringia		&	5.318		&	3.407	&	-0.572	& 0.968	&	645		\\
			\midrule
			Total			&	19.676		&	8.234	&	9.113	& 4.325	&	12,191	\\
			\bottomrule
		\end{tabular*}
		\begin{tablenotes} 
			\item \scriptsize \emph{Notes:} The Table reports the population-weighted mean and standard deviation for the Green's vote share in 2019 (columns 2 and 3) and the first-differenced vote share for the Greens (columns 4 and 5) across the 2019 elections. The last column reports the number of municipalities per election. 
		\end{tablenotes} 
	\end{threeparttable}
\end{table}