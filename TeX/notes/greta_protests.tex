% 	TO DO: 
%	MZ Tabelle nicht einfügen: DDRD estimate von Regression mit 3 CG


%--------------------------------------------------------------------
%	DOCUMENT CLASS
%--------------------------------------------------------------------
\documentclass[11pt, a4paper]{article} % type of document (paper, presentation, book,...); scrartcl class with sans serif titles, European layout 
\usepackage{fullpage} % leaves less space at margins of page
\usepackage[onehalfspacing]{setspace} % determine line pitch to 1.5

%--------------------------------------------------------------------
%	INPUT
%--------------------------------------------------------------------
\usepackage[T1]{fontenc} 	% Use 8-bit encoding that has 256 glyphs
\usepackage[utf8]{inputenc} % Required for including letters with accents, Umlaute,...
\usepackage{float} 			% better control over placement of tables and figures in the text
\usepackage{graphicx} 		% input of graphics
\usepackage{xcolor} 		% advanced color package
\usepackage{url} 			% include (clickable) URLs
\usepackage[breaklinks=true]{hyperref}
\usepackage{pdfpages}		% insert pages of external pdf documents
\setlength{\parskip}{0em}	% vertical spacing for paragraphs
\setlength{\parindent}{0em}	% horizonzal spacing for paragraphs
\usepackage{tikz}
\usepackage{tikzscale}		% helps to adjust tikz pictures to textwidth/linewidth
\usetikzlibrary{decorations.pathreplacing}
\usetikzlibrary{patterns}
\usetikzlibrary{arrows}
\usepackage{eurosym}		% Eurosymbol

% Have sections in TOC, but not in text
\usepackage{xparse}% for easier management of optional arguments
\ExplSyntaxOn
\NewDocumentCommand{\TODO}{msom}
{
	\IfBooleanF{#1}% do nothing if it's starred
	{
		\cs_if_eq:NNT #1 \chapter { \cleardoublepage\mbox{} }
		\refstepcounter{\cs_to_str:N #1}
		\IfNoValueTF{#3}
		{
			\addcontentsline{toc}{\cs_to_str:N #1}{\protect\numberline{\use:c{the\cs_to_str:N #1}}#4}
		}
		{
			\addcontentsline{toc}{\cs_to_str:N #1}{\protect\numberline{\use:c{the\cs_to_str:N #1}}#3}
		}
	}
	\cs_if_eq:NNF #1 \chapter { \mbox{} }% allow page breaks after sections
}
\ExplSyntaxOff

%--------------------------------------------------------------------
%	TABLES, FIGURES, LISTS
%--------------------------------------------------------------------
\usepackage{booktabs} 		% better tables
\usepackage{longtable}		% tables that may be continued on the next page
\usepackage{threeparttable} % add notes below tables
\renewcommand\TPTrlap{}		% add margins on the side of the notes
	\renewcommand\TPTnoteSettings{%
	\setlength\leftmargin{5 pt}%
	\setlength\rightmargin{5 pt}%
}
\usepackage[
center, format=plain,
font=normalsize,
nooneline,
labelfont={bf}
]{caption} 				% change format of captions of tables and graphs 
%USED IN MPHIL: \usepackage[labelfont=bf,labelsep = period, singlelinecheck=off,justification=raggedright]{caption}, other specifications which are nice: labelformat = parens -> number in paranthesis 


%\usepackage{threeparttablex} % for "ThreePartTable" environment, helps to combine threepart and longtable

% Allow line breaks with \\ in column headings of tables
\newcommand{\clb}[3][c]{%
	\begin{tabular}[#1]{@{}#2@{}}#3\end{tabular}}

% allow line breaks with \\ in row titles
\usepackage{multirow}

\newcommand{\rlb}[3][c]{%
\multirow{2}{*}{\begin{tabular}[#1]{@{}#2@{}}#3\end{tabular}}}% optional argument: b = bottom or t= top alignment


\usepackage[singlelinecheck=on]{subcaption}%both together help to have subfigures
\usepackage{wrapfig}				% wrap text around figure


\usepackage{rotating}				% rotating figures & tables
\usepackage{enumerate}				% change appearance of the enumerator
\usepackage{paralist, enumitem}		% better enumerations
\setlist{noitemsep}					% no additional vertical spacing for enurations
%--------------------------------------------------------------------
%	MATH
%--------------------------------------------------------------------
\usepackage{amsmath,amssymb,amsfonts} % more math symbols and commands
\let\vec\mathbf				 % make vector bold, with no arrow and not in italic

%--------------------------------------------------------------------
%	LANGUAGE SPECIFICS
%--------------------------------------------------------------------
\usepackage[american]{babel} % man­ages cul­tur­ally-de­ter­mined ty­po­graph­i­cal (and other) rules, and hy­phen­ation pat­terns
\usepackage{csquotes} % language specific quotations

%--------------------------------------------------------------------
%	BIBLIOGRAPHY & CITATIONS
%--------------------------------------------------------------------
\usepackage{csquotes} % language specific quotations
\usepackage{etex}		% some more Tex functionality
\usepackage[nottoc]{tocbibind} %add bibliography to TOC
\usepackage[authoryear, round, comma]{natbib} %biblatex

%--------------------------------------------------------------------
%	PATHS
%--------------------------------------------------------------------
\makeatletter
\def\input@path{{W:/EoCC/analysis/output/graphs/descriptive/}}  	%PATH TO TABLES
%or: \def\input@path{{/path/to/folder/}{/path/to/other/folder/}}
\makeatother
\graphicspath{{W:/EoCC/analysis/output/graphs/descriptive/}}		% PATH TO GRAPHS

%--------------------------------------------------------------------
%	LAYOUT
%--------------------------------------------------------------------
\usepackage[left=3cm,right=3cm,top=2cm,bottom=3cm]{geometry}
\usepackage{pdflscape} % lscape.sty Produce landscape pages in a (mainly) portrait document.

\definecolor{darkblue}{rgb}{0.0,0.0,0.6}
\newcommand\natalia[1]{\textcolor{orange}{#1}}

% CAPTIAL LETTERS FOR SECTION CAPTIONS
%\usepackage{sectsty}
%\sectionfont{\normalfont\scshape\centering\textbf}
%\renewcommand{\thesection}{\Roman{section}.}
%\renewcommand{\thesubsection}{\Alph{subsection}.}%\thesection\Alph{subsection}.
%\subsectionfont{\itshape}
%\subsubsectionfont{\scshape}
%\newcommand\relphantom[1]{\mathrel{\phantom{#1}}}
%\setlength\topmargin{0.1in} \setlength\headheight{0.1in}
%\setlength\headsep{0in} \setlength\textheight{9.2in}
%\setlength\textwidth{6.3in} \setlength\oddsidemargin{0.1in}
%\setlength\evensidemargin{0.1in}

\hypersetup{
  colorlinks  = true,
  citecolor   = darkblue,
 	linkcolor   = darkblue,
  urlcolor    = darkblue 
} % macht die URLS blau   
     
\usepackage{lettrine}	% First letter capitalized

% have date in month year format (i.e. omit the day in dates)
\usepackage{datetime}
\newdateformat{monthyeardate}{%
  \monthname[\THEMONTH], \THEYEAR}
%--------------------------------------------------------------------
%	AUTHOR & TITLE
%--------------------------------------------------------------------
\title{The effect of Greta Thunberg on consumption behavior}
\author{Marc Fabel, Helmut Rainer, Maria Waldinger, Sebastian Wichert}

\date{\monthyeardate\today}








%--------------------------------------------------------------------
%	BEGIN DOCUMENT
%--------------------------------------------------------------------




\begin{document}
\setcounter{page}{0}  
% \tableofcontents
\newpage
\setcounter{page}{1}    
\maketitle

%\textbf{\color{red} Preliminary and incomplete draft\newline Please do not cite or circulate without the author's permission}
%\renewcommand{\abstractname}{\vspace{-\baselineskip}} % GET RID OF ABSTRACT TITLE

%  \begin{abstract}\noindent 
%   \footnotesize{\begin{center}\textbf{Abstract}\end{center} Place abstract here}
%    \end{abstract}

\bigskip
\tableofcontents

\newpage


%--------------------------------------------------------------------
% INTRODUCTION
%--------------------------------------------------------------------
\section{Introduction}\label{sec_greta_cons:introduction}






%--------------------------------------------------------------------
% BACKGROUND
%--------------------------------------------------------------------
\bigskip
\section{Background}\label{sec_greta_cons:background}

\subsection{The phenomenon Greta Thunberg}

\textbf{Timeline (important dates)}
\begin{itemize}
	\item 20.08.2018 First time in front of Swedish Parliament
	\item 27.08.2018 First coverage in German newspaper
	\item 3.12.2018-14.12.2018 Katowice (climate conference )
	\item mid December first climate strikes in Germany
	\item 23.01.2019-25.01.2019 Davos (world economic forum)
	\item 01.03.2019 Greta visits Hamburg for climate strike
	\item 15.03.2019 Worldwide climate strikes, more than 1.4 million people involved
	\item 29.03.2019 Greta visits Berlin for climate strike
	\item 16.04.2019 Strasbourg speech at EU
	\item 24.05.2019 2nd Global Climate Strike (for EU elections)
	\item 21.06.2019 Aachen: Climate Justice w/o borders
	\item 14.08.-28.08.2019 Journey across Atlantic 
	\item 20.09.-27.09.2019 Global Week of Climate Action
	\item 23.09.-29.09.2019 NY - UN climate action summit
	\item 29.11.2019 Fourth Global Climate Strike
	\item 02.12.-13.12.2019 Madrid (UN climate change conference )
	\item 11.12.2019 Greta Thunberg Time Person of the Year	
\end{itemize}
see coverage of Greta Thunberg in print media (daily and weekly outlets) in Figures \ref{fig_greta_cons:genios_greta_per_1000_2019} and \ref{fig_greta_cons:genios_greta_per_1000_events_2019}

%--------------------------------------------------------------------
% IDENTIFICATION
%--------------------------------------------------------------------
\newpage
\section{Empirical strategy}\label{sec_greta_cons:empirical_strategy}





%--------------------------------------------------------------------
% DATA & VARIABLES
%--------------------------------------------------------------------
\newpage
\section{Data}\label{sec_greta_cons:data} 













%--------------------------------------------------------------------
% RESULTS
%--------------------------------------------------------------------
\newpage
\section{Results}\label{sec_greta_cons:results}













%--------------------------------------------------------------------
% CONCLUSION
%-------------------------------------------------------------------
\bigskip
\section{Concluding remarks}\label{sec_greta_cons:conclusion}




%--------------------------------------------------------------------
% BIBLIOGRAPHY
%--------------------------------------------------------------------
\newpage


%\bibliographystyle{ecca_edited}%previous style-chicago
%\bibliography{greta_cons_bibliography}




%--------------------------------------------------------------------
% FIGURES AND TABLES
%--------------------------------------------------------------------
%\newpage
%\section{Figures and tables}
\newpage 
\par
\section{Figures}
\vspace*{\fill}
{\Huge \begin{center}\textbf{FIGURES}\end{center}}
\vspace*{\fill}\clearpage
%--------------------------------------------


%--------------------------------------------
% Genios Greta Thunberg
\begin{figure}[H]\centering
	\caption{Number of protests over time}\label{fig_protests:protests}
	\includegraphics[width=0.9\linewidth]{W:/EoCC/analysis/output/graphs/descriptive/protest.pdf}
	\begin{minipage}{\linewidth}
		\scriptsize{\emph{Notes:} Number of protests above 130 are displayed}
	\end{minipage}
\end{figure}




\begin{figure}[H]\centering
	\caption{Map of the number of protests by state}\label{fig_protests:map}
	\includegraphics[width=0.9\linewidth]{W:/EoCC/analysis/output/graphs/descriptive/protests_by_state_map_labels.pdf}	
\end{figure}




\begin{figure}[H]\centering
	\caption{Number of protests by state}\label{fig_protests:map}
	\includegraphics[width=0.9\linewidth]{W:/EoCC/analysis/output/graphs/descriptive/protests_by_state.pdf}	
\end{figure}


\section{Registered participants in demonstrations}

\begin{figure}[H]\centering
	\caption{Number registered participants by municipality}\label{registered_Gemeinde}
	\includegraphics[width=0.9\linewidth]{W:/EoCC/analysis/output/graphs/descriptive/Gemeinde.pdf}	
\end{figure}


\begin{figure}[H]\centering
	\caption{Number registered participants by county}\label{registered_Kreis}
	\includegraphics[width=0.9\linewidth]{W:/EoCC/analysis/output/graphs/descriptive/Kreis.pdf}	
\end{figure}


\begin{figure}[H]\centering
	\caption{Number registered participants by city}\label{registered_city}
	\includegraphics[width=0.9\linewidth]{W:/EoCC/analysis/output/graphs/descriptive/map_registered.pdf}	
\end{figure}

\end{document}