% 	TO DO: 
%	MZ Tabelle nicht einfügen: DDRD estimate von Regression mit 3 CG


%--------------------------------------------------------------------
%	DOCUMENT CLASS
%--------------------------------------------------------------------
\documentclass[11pt, a4paper]{article} % type of document (paper, presentation, book,...); scrartcl class with sans serif titles, European layout 
\usepackage{fullpage} % leaves less space at margins of page
\usepackage[onehalfspacing]{setspace} % determine line pitch to 1.5

%--------------------------------------------------------------------
%	INPUT
%--------------------------------------------------------------------
\usepackage[T1]{fontenc} 	% Use 8-bit encoding that has 256 glyphs
\usepackage[utf8]{inputenc} % Required for including letters with accents, Umlaute,...
\usepackage{float} 			% better control over placement of tables and figures in the text
\usepackage{graphicx} 		% input of graphics
\usepackage{xcolor} 		% advanced color package
\usepackage{url} 			% include (clickable) URLs
\usepackage[breaklinks=true]{hyperref}
\usepackage{pdfpages}		% insert pages of external pdf documents
\setlength{\parskip}{0em}	% vertical spacing for paragraphs
\setlength{\parindent}{0em}	% horizonzal spacing for paragraphs
\usepackage{tikz}
\usepackage{tikzscale}		% helps to adjust tikz pictures to textwidth/linewidth
\usetikzlibrary{decorations.pathreplacing}
\usetikzlibrary{patterns}
\usetikzlibrary{arrows}
\usepackage{eurosym}		% Eurosymbol

% Have sections in TOC, but not in text
\usepackage{xparse}% for easier management of optional arguments
\ExplSyntaxOn
\NewDocumentCommand{\TODO}{msom}
{
	\IfBooleanF{#1}% do nothing if it's starred
	{
		\cs_if_eq:NNT #1 \chapter { \cleardoublepage\mbox{} }
		\refstepcounter{\cs_to_str:N #1}
		\IfNoValueTF{#3}
		{
			\addcontentsline{toc}{\cs_to_str:N #1}{\protect\numberline{\use:c{the\cs_to_str:N #1}}#4}
		}
		{
			\addcontentsline{toc}{\cs_to_str:N #1}{\protect\numberline{\use:c{the\cs_to_str:N #1}}#3}
		}
	}
	\cs_if_eq:NNF #1 \chapter { \mbox{} }% allow page breaks after sections
}
\ExplSyntaxOff

%--------------------------------------------------------------------
%	TABLES, FIGURES, LISTS
%--------------------------------------------------------------------
\usepackage{booktabs} 		% better tables
\usepackage{longtable}		% tables that may be continued on the next page
\usepackage{threeparttable} % add notes below tables
\renewcommand\TPTrlap{}		% add margins on the side of the notes
	\renewcommand\TPTnoteSettings{%
	\setlength\leftmargin{5 pt}%
	\setlength\rightmargin{5 pt}%
}
\usepackage[
center, format=plain,
font=normalsize,
nooneline,
labelfont={bf}
]{caption} 				% change format of captions of tables and graphs 
%USED IN MPHIL: \usepackage[labelfont=bf,labelsep = period, singlelinecheck=off,justification=raggedright]{caption}, other specifications which are nice: labelformat = parens -> number in paranthesis 


%\usepackage{threeparttablex} % for "ThreePartTable" environment, helps to combine threepart and longtable

% Allow line breaks with \\ in column headings of tables
\newcommand{\clb}[3][c]{%
	\begin{tabular}[#1]{@{}#2@{}}#3\end{tabular}}

% allow line breaks with \\ in row titles
\usepackage{multirow}

\newcommand{\rlb}[3][c]{%
\multirow{2}{*}{\begin{tabular}[#1]{@{}#2@{}}#3\end{tabular}}}% optional argument: b = bottom or t= top alignment


\usepackage[singlelinecheck=on]{subcaption}%both together help to have subfigures
\usepackage{wrapfig}				% wrap text around figure


\usepackage{rotating}				% rotating figures & tables
\usepackage{enumerate}				% change appearance of the enumerator
\usepackage{paralist, enumitem}		% better enumerations
\setlist{noitemsep}					% no additional vertical spacing for enurations
%--------------------------------------------------------------------
%	MATH
%--------------------------------------------------------------------
\usepackage{amsmath,amssymb,amsfonts} % more math symbols and commands
\let\vec\mathbf				 % make vector bold, with no arrow and not in italic

%--------------------------------------------------------------------
%	LANGUAGE SPECIFICS
%--------------------------------------------------------------------
\usepackage[american]{babel} % man­ages cul­tur­ally-de­ter­mined ty­po­graph­i­cal (and other) rules, and hy­phen­ation pat­terns
\usepackage{csquotes} % language specific quotations

%--------------------------------------------------------------------
%	BIBLIOGRAPHY & CITATIONS
%--------------------------------------------------------------------
\usepackage{csquotes} % language specific quotations
\usepackage{etex}		% some more Tex functionality
\usepackage[nottoc]{tocbibind} %add bibliography to TOC
\usepackage[authoryear, round, comma]{natbib} %biblatex

%--------------------------------------------------------------------
%	PATHS
%--------------------------------------------------------------------
\makeatletter
\def\input@path{{../../analysis/output/tables/}}	%PATH TO TABLES
%or: \def\input@path{{/path/to/folder/}{/path/to/other/folder/}}
\makeatother
\graphicspath{{../../analysis/output/graphs/}}		% PATH TO GRAPHS

%--------------------------------------------------------------------
%	LAYOUT
%--------------------------------------------------------------------
\usepackage[left=3cm,right=3cm,top=2cm,bottom=3cm]{geometry}
\usepackage{pdflscape} % lscape.sty Produce landscape pages in a (mainly) portrait document.

\definecolor{darkblue}{rgb}{0.0,0.0,0.6}
\newcommand\natalia[1]{\textcolor{orange}{#1}}

% CAPTIAL LETTERS FOR SECTION CAPTIONS
%\usepackage{sectsty}
%\sectionfont{\normalfont\scshape\centering\textbf}
%\renewcommand{\thesection}{\Roman{section}.}
%\renewcommand{\thesubsection}{\Alph{subsection}.}%\thesection\Alph{subsection}.
%\subsectionfont{\itshape}
%\subsubsectionfont{\scshape}
%\newcommand\relphantom[1]{\mathrel{\phantom{#1}}}
%\setlength\topmargin{0.1in} \setlength\headheight{0.1in}
%\setlength\headsep{0in} \setlength\textheight{9.2in}
%\setlength\textwidth{6.3in} \setlength\oddsidemargin{0.1in}
%\setlength\evensidemargin{0.1in}

\hypersetup{
  colorlinks  = true,
  citecolor   = darkblue,
 	linkcolor   = darkblue,
  urlcolor    = darkblue 
} % macht die URLS blau   
     
\usepackage{lettrine}	% First letter capitalized

% have date in month year format (i.e. omit the day in dates)
\usepackage{datetime}
\newdateformat{monthyeardate}{%
  \monthname[\THEMONTH], \THEYEAR}
%--------------------------------------------------------------------
%	AUTHOR & TITLE
%--------------------------------------------------------------------
\title{Descriptive Graphs about Greta Thunberg \& FFF}
\author{Marc Fabel}

\date{\monthyeardate\today}







%--------------------------------------------------------------------
%	BEGIN DOCUMENT
%--------------------------------------------------------------------
\begin{document}
\maketitle
The document contains figures that depict the evolution of the phenomenon Greta Thunberg or FFF. The first half is dedicated to the evolution in print media, whereas the second half is covering data obtained from Greta Thunberg's Twitter feed.\newline

Data which is not presented here, but we still have: 
\begin{itemize}
	\item the number of articles (about Greta Thunberg, FFF, ...) across all outlets that are listed on genios
\end{itemize}

\tableofcontents


\newpage
\TODO\section{Print Media}
\vspace*{\fill}
{\Huge \begin{center}\textbf{Print Media}\end{center}}
\vspace*{\fill}\clearpage
%--------------------------------------------


%--------------------------------------------------------------------
% Greta Thunberg
%--------------------------------------------------------------------
\subsection{Search results in Genios for the term "Greta Thunberg"}


\begin{figure}[H]\centering
	\caption{Time series of the number of articles about Greta Thunberg, 2018-2019}
	\includegraphics[width=0.9\linewidth]{descriptive/greta_cons_genios_greta_2018_2019}
	\begin{minipage}{\linewidth}
		\scriptsize{\emph{Notes:} }
	\end{minipage}
\end{figure}
%--------------
\begin{figure}[H]\centering
	\caption{Time series of the number of articles about Greta Thunberg per 1,000 articles, 2018-2019}
	\includegraphics[width=0.9\linewidth]{descriptive/greta_cons_genios_greta_per_1000_2018_2019}
	\begin{minipage}{\linewidth}
		\scriptsize{\emph{Notes:} The orange line is from a moving average of 7 days}
	\end{minipage}
\end{figure}
%--------------

\begin{figure}[H]\centering
	\caption{Time series of the number of articles about Greta Thunberg per 1,000 articles, 2019}
	\includegraphics[width=0.9\linewidth]{descriptive/greta_cons_genios_greta_per_1000_2019}
	\begin{minipage}{\linewidth}
		\scriptsize{\emph{Notes:} The blue line is from a moving average with window of 3 days}
	\end{minipage}
\end{figure}


\begin{figure}[H]\centering
	\caption{Time series of the number of articles about Greta Thunberg per 1,000 articles, 2019}
	\includegraphics[width=0.9\linewidth]{descriptive/greta_cons_genios_greta_per_1000_events_2019}
	\begin{minipage}{\linewidth}
		\scriptsize{\emph{Notes:} The blue line is from a moving average with window of 3 days. The green lines/areas indicate important events: 1) Katowice (climate conference 3.12.2018-14.12.2018), 2) Davos (world economic forum 23.01.2019-25.01.2019), 3) Hamburg (climate strike 01.03.2019), 4) Worldwide (climate strikes, more than 1.4 million people involved 15.03.2019), 5) Berlin (climate strike 29.03.2019), 6) Strasbourg (speech EU 16.04.2019), 7) 2nd Global Climate Strike(24.05.2019),	8) Journey across Atlantic (14.08.-28.08.2019) 9) NY (UN climate action summit 23.09.-29.09.2019), 10) Madrid (UN climate change conference 02.12.-13.12.2019).}
	\end{minipage}
\end{figure}


\begin{figure}[H]\centering
	\caption{The evolution of Greta Thunberg in the two leading German newspapers}
	\includegraphics[width=0.9\linewidth]{descriptive/greta_cons_genios_greta_SZ_FAZ}
	\begin{minipage}{\linewidth}
		\scriptsize{\emph{Notes:} The figure shows the weekly number of articles covering Greta Thunberg.}
	\end{minipage}
\end{figure}
\begin{figure}[H]\centering
	\caption{The evolution of Greta Thunberg in the two leading German newspapers}
	\includegraphics[width=0.9\linewidth]{descriptive/greta_cons_genios_greta_SZ_FAZ_all_newspapers}
	\begin{minipage}{\linewidth}
		\scriptsize{\emph{Notes:} The figure shows the weekly number of articles covering Greta Thunberg.}
	\end{minipage}
\end{figure}


%--------------------------------------------------------------------
% Fridays For Future
%--------------------------------------------------------------------
\newpage
\subsection{Search results in Genios for terms covering the topic Fridays for Future}


\begin{figure}[H]\centering
	\caption{Time series of the number of articles about FFF per 1,000 articles, 2019}
	\includegraphics[width=0.85\linewidth]{descriptive/greta_cons_genios_fff_per_1000_2019}
	\begin{minipage}{\linewidth}
		\scriptsize{\emph{Notes:} The blue line is from a moving average with window of 3 days.}
	\end{minipage}
\end{figure}


\begin{figure}[H]\centering
	\caption{Time series of the number of articles about FFF per 1,000 articles, 2019}
	\includegraphics[width=0.85\linewidth]{descriptive/greta_cons_genios_fff_per_1000_events_2019}
	\begin{minipage}{\linewidth}
		\scriptsize{\emph{Notes:} The blue line is from a moving average with window of 3 days. The green lines/areas indicate important events: 1) Davos (world economic forum 23.01.2019-25.01.2019), 2) Hamburg (climate strike 01.03.2019), 3) Worldwide (climate strikes, more than 1.4 million people involved 15.03.2019), 4) Berlin (climate strike 29.03.2019), 5) 2nd Global Climate Strike(24.05.2019), (21.06.2019), 6) Global Week of Climate Action (20.-27.09.2019), 7) Fourth Global Climate Strike(29.11.2019)}
	\end{minipage}
\end{figure}



%--------------------------------------------------------------------
% TWITTER
%--------------------------------------------------------------------
\newpage
\TODO\section{Twitter Data}
\vspace*{\fill}
{\Huge \begin{center}\textbf{Twitter Data}\end{center}}
\vspace*{\fill}\clearpage
%--------------------------------------------
\subsection{Greta Thunberg's Twitter feed}
\begin{figure}[H]\centering
	\caption{The evolution of Greta Thunberg's Twitter feed}
	\includegraphics[width=0.85\linewidth]{descriptive/greta_cons_twitter_greta_favorites_retweets_weekly}
	\begin{minipage}{\linewidth}
		\scriptsize{\emph{Notes:} The figure plots the weekly number of favorites and retweets (in thousand).}
	\end{minipage}
\end{figure}


\begin{figure}[H]\centering
	\caption{The evolution of Greta Thunberg's Twitter feed}
	\includegraphics[width=0.85\linewidth]{descriptive/greta_cons_twitter_greta_favorites_counts_weekly}
	\begin{minipage}{\linewidth}
		\scriptsize{\emph{Notes:} The figure plots the weekly number of tweets and favorites (in thousand).}
	\end{minipage}
\end{figure}

\begin{figure}[H]\centering
	\caption{The evolution of Greta Thunberg's Twitter feed}
	\includegraphics[width=0.85\linewidth]{descriptive/greta_cons_twitter_greta_favorites_retweets_weekly_events}
	\begin{minipage}{\linewidth}
		\scriptsize{\emph{Notes:} The figure plots the weekly number of favorites and retweets (in thousand).The green lines/areas indicate important events: 1) Katowice (climate conference 3.12.2018-14.12.2018), 2) Davos (world economic forum 23.01.2019-25.01.2019), 3) Hamburg (climate strike 01.03.2019), 4) Worldwide (climate strikes, more than 1.4 million people involved 15.03.2019), 5) Berlin (climate strike 29.03.2019), 6) Strasbourg (speech EU 16.04.2019), 7) Journey across Atlantic (14.08.-28.08.2019) 8) NY (UN climate action summit 23.09.-29.09.2019), 9) Madrid (UN climate change conference 02.12.-13.12.2019).}
	\end{minipage}
\end{figure}



\newpage
\begin{figure}[H]\centering
	\caption{Word Cloud of Greta Thunberg's tweets}
	\includegraphics[width=0.85\linewidth]{descriptive/greta_cons_twitter_greta_word_cloud.pdf}
	\begin{minipage}{\linewidth}
		\scriptsize{\emph{Notes:} The figure shows the contents of Greta Thunberg's tweets. The larger one word is depicted, the more frequent it is used by Greta Thunberg in her tweets. The lowercased keywords are from the subset of words after removing urls, hastags, mentions, digits, punctuation, and stopwords.}
	\end{minipage}
\end{figure}

\begin{figure}[H]\centering
	\caption{Top words in Greta Thunberg's tweets}
	\includegraphics[width=0.95\linewidth]{descriptive/greta_cons_twitter_greta_frequency_common_words.pdf}
	\begin{minipage}{\linewidth}
		\scriptsize{\emph{Notes:} The figure shows the frequency of common words in Greta Thunberg's tweets.}
	\end{minipage}
\end{figure}


\begin{figure}[H]\centering
	\caption{Top hashtags in Greta Thunberg's tweets}
	\includegraphics[width=0.95\linewidth]{descriptive/greta_cons_twitter_greta_frequency_common_hashtags.pdf}
	\begin{minipage}{\linewidth}
		\scriptsize{\emph{Notes:} The figure shows the frequency of the most widely used hashtags in Greta Thunberg's tweets. The hashtags are listed if they appear at least five times over the entire time period. }
	\end{minipage}
\end{figure}


\begin{figure}[H]\centering
	\caption{Top mentions in Greta Thunberg's tweets}
	\includegraphics[width=0.95\linewidth]{descriptive/greta_cons_twitter_greta_frequency_common_mentions.pdf}
	\begin{minipage}{\linewidth}
		\scriptsize{\emph{Notes:}The figure lists the ten most frequent mentions in Greta Thunberg's tweets. The people behind the Twitter accounts are: @$\_$NikkiHenderson, @Sailing$\_$LaVaga, @elayna$\_$c (sailors), @GeorgeMonbiot (British writer and politcal activist),  @KevinClimate (Professor of energy and climate change), @Luisamneubauer (German climate activist), @AnunaDe (Belgian climate activist), @ExtinctionR (global environmental movement), @UNFCCC (UN Climate Change), @KlimatSverige (Swedish non-profit climate activism network)}
	\end{minipage}
\end{figure}




\begin{figure}[H]\centering
	\caption{Conditional sentiment analysis of Greta Thunberg's tweets}
	\begin{subfigure}[h]{0.48\linewidth}\centering\caption{Polarity}
		\includegraphics[width=\linewidth]{descriptive/greta_cons_twitter_greta_polarity_hist.pdf}
	\end{subfigure}
	\begin{subfigure}[h]{0.48\linewidth}\centering\caption{Subjectivity}
		\includegraphics[width=\linewidth]{descriptive/greta_cons_twitter_greta_subjectivity_hist.pdf}
	\end{subfigure}
	\begin{minipage}{\linewidth}
		\scriptsize{\emph{Notes:} The histograms show the conditional distribution of tweets in the dimensions of polarity and subjectivity. We can see that there are more tweets with polarity values larger than zero, indicating more positive tweets. Furthermore, Greta Thunberg's tweets tend to be formulated more subjectively. We use the \texttt{TextBlob} library in order to classify the sentiments of the tweets.}
	\end{minipage}
\end{figure}



%--------------------------------------------------------------------
% TWITTER - OTHER FFF ICONS
%--------------------------------------------------------------------
\newpage
\subsection{The Twitter feed of other famous FFF icons}

\begin{figure}[H]\centering
	\caption{The polarity and subjectivity of other FFF icons}
	\includegraphics[width=\linewidth]{descriptive/greta_cons_twitter_activists_scatter_polarity_subjectivity}
	\begin{minipage}{\linewidth}
		\scriptsize{\emph{Notes:} The figure shows the conditional means of the tweets' polarity and subjectivity across important FFF activists.}
	\end{minipage}
\end{figure}


\newpage
\begin{landscape}

	\begin{figure}[H]\centering
		\caption{Twitter feeds of important FFF icons - Number of favorites across different activists}
		\begin{subfigure}[h]{0.23\linewidth}\centering\caption{All}
			\includegraphics[width=\linewidth]{descriptive/greta_cons_twitter_favorites_spaghetti_all.pdf}
		\end{subfigure}
		\begin{subfigure}[h]{0.23\linewidth}\centering\caption{Luisa Neubauer}
			\includegraphics[width=\linewidth]{descriptive/greta_cons_twitter_favorites_spaghetti_Luisamneubauer.pdf}
		\end{subfigure}
		\begin{subfigure}[h]{0.23\linewidth}\centering\caption{Jakob Blasel}
			\includegraphics[width=\linewidth]{descriptive/greta_cons_twitter_favorites_spaghetti_jakobblasel.pdf}
		\end{subfigure}
		\begin{subfigure}[h]{0.23\linewidth}\centering\caption{Carla Reemtsma}
			\includegraphics[width=\linewidth]{descriptive/greta_cons_twitter_favorites_spaghetti_carla_reemtsma.pdf}
		\end{subfigure}
	
		\begin{subfigure}[h]{0.23\linewidth}\centering\caption{Franzi Wessel}
			\includegraphics[width=\linewidth]{descriptive/greta_cons_twitter_favorites_spaghetti_FranziWessel.pdf}
		\end{subfigure}
		\begin{subfigure}[h]{0.23\linewidth}\centering\caption{FridayForFuture Germany}
			\includegraphics[width=\linewidth]{descriptive/greta_cons_twitter_favorites_spaghetti_FridayForFuture.pdf}
		\end{subfigure}
		\begin{subfigure}[h]{0.23\linewidth}\centering\caption{Ende Gelaende}
			\includegraphics[width=\linewidth]{descriptive/greta_cons_twitter_favorites_spaghetti_Ende__Gelaende.pdf}
		\end{subfigure}
		\begin{subfigure}[h]{0.23\linewidth}\centering\caption{parents4future}
			\includegraphics[width=\linewidth]{descriptive/greta_cons_twitter_favorites_spaghetti_parents4future.pdf}
		\end{subfigure}
	
		\begin{subfigure}[h]{0.23\linewidth}\centering\caption{Extinction Rebellion}
			\includegraphics[width=\linewidth]{descriptive/greta_cons_twitter_favorites_spaghetti_ExtinctionR_DE.pdf}
		\end{subfigure}
		\begin{subfigure}[h]{0.23\linewidth}\centering\caption{sciforfuture}
			\includegraphics[width=\linewidth]{descriptive/greta_cons_twitter_favorites_spaghetti_sciforfuture.pdf}
		\end{subfigure}
		\begin{minipage}{\linewidth}
			\scriptsize{\emph{Notes:} The figures show the number of favorites [in thousand] across different FFF icons.}
		\end{minipage}
	\end{figure}

\end{landscape}




\begin{landscape}
	\begin{figure}[H]\centering
		\caption{Twitter feeds of important FFF icons}
		\begin{subfigure}[h]{0.23\linewidth}\centering\caption{Greta Thunberg}
			\includegraphics[width=\linewidth]{descriptive/greta_cons_twitter_favorites_retweets_GretaThunberg.pdf}
		\end{subfigure}
		\begin{subfigure}[h]{0.23\linewidth}\centering\caption{Luisa Neubauer}
			\includegraphics[width=\linewidth]{descriptive/greta_cons_twitter_favorites_retweets_Luisamneubauer.pdf}
		\end{subfigure}
		\begin{subfigure}[h]{0.23\linewidth}\centering\caption{Jakob Blasel}
			\includegraphics[width=\linewidth]{descriptive/greta_cons_twitter_favorites_retweets_jakobblasel.pdf}
		\end{subfigure}
		\begin{subfigure}[h]{0.23\linewidth}\centering\caption{Carla Reemtsma}
			\includegraphics[width=\linewidth]{descriptive/greta_cons_twitter_favorites_retweets_carla_reemtsma.pdf}
		\end{subfigure}
		
		\begin{subfigure}[h]{0.23\linewidth}\centering\caption{Franzi Wessel}
			\includegraphics[width=\linewidth]{descriptive/greta_cons_twitter_favorites_retweets_FranziWessel.pdf}
		\end{subfigure}
		\begin{subfigure}[h]{0.23\linewidth}\centering\caption{FridayForFuture Germany}
			\includegraphics[width=\linewidth]{descriptive/greta_cons_twitter_favorites_retweets_FridayForFuture.pdf}
		\end{subfigure}
		\begin{subfigure}[h]{0.23\linewidth}\centering\caption{Ende Gelaende}
			\includegraphics[width=\linewidth]{descriptive/greta_cons_twitter_favorites_retweets_Ende__Gelaende.pdf}
		\end{subfigure}
		\begin{subfigure}[h]{0.23\linewidth}\centering\caption{parents4future}
			\includegraphics[width=\linewidth]{descriptive/greta_cons_twitter_favorites_retweets_parents4future.pdf}
		\end{subfigure}
		
		\begin{subfigure}[h]{0.23\linewidth}\centering\caption{Extinction Rebellion}
			\includegraphics[width=\linewidth]{descriptive/greta_cons_twitter_favorites_retweets_ExtinctionR_DE.pdf}
		\end{subfigure}
		\begin{subfigure}[h]{0.23\linewidth}\centering\caption{sciforfuture}
			\includegraphics[width=\linewidth]{descriptive/greta_cons_twitter_favorites_retweets_sciforfuture.pdf}
		\end{subfigure}
		\begin{minipage}{\linewidth}
			\scriptsize{\emph{Notes:} The figures show the number of favorites and retweets [in thousand] across different FFF icons.}
		\end{minipage}
	\end{figure}	
\end{landscape}





\end{document}