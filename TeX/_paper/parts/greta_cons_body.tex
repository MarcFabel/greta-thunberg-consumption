literature: 
\begin{itemize}
	\item mobile-phone based tracking data: \cite{dave2020contagion}
\end{itemize}



%--------------------------------------------------------------------
% BACKGROUND
%--------------------------------------------------------------------
\bigskip
\section{Background}\label{sec_greta_cons:background}



\textbf{Timeline (important dates)}
\begin{itemize}
	\item 20.08.2018 First time in front of Swedish Parliament
	\item 27.08.2018 First coverage in German newspaper
	\item 3.12.2018-14.12.2018 Katowice (climate conference )
	\item mid December first climate strikes in Germany
	\item 23.01.2019-25.01.2019 Davos (world economic forum)
	\item 01.03.2019 Greta visits Hamburg for climate strike
	\item 15.03.2019 Worldwide climate strikes, more than 1.4 million people involved
	\item 29.03.2019 Greta visits Berlin for climate strike
	\item 16.04.2019 Strasbourg speech at EU
	\item 24.05.2019 2nd Global Climate Strike (for EU elections)
	\item 21.06.2019 Aachen: Climate Justice w/o borders
	\item 14.08.-28.08.2019 Journey across Atlantic 
	\item 20.09.-27.09.2019 Global Week of Climate Action
	\item 23.09.-29.09.2019 NY - UN climate action summit
	\item 29.11.2019 Fourth Global Climate Strike
	\item 02.12.-13.12.2019 Madrid (UN climate change conference )
	\item 11.12.2019 Greta Thunberg Time Person of the Year	
\end{itemize}
see coverage of Greta Thunberg in print media (daily and weekly outlets) in Figures \ref{fig_greta_cons:genios_greta_per_1000_2019} and \ref{fig_greta_cons:genios_greta_per_1000_events_2019}





% notes 
movement initiated by Greta Thunberg, who initiated them in August 2018
she stands as noone else as 

school climate strike movement (FFF)
`The Greta Effect' (e.g. The Guardian )Influence on the world stage

prizes 
two consecutive nominations for the Nobel Peace Prize (2019 \& 2020)







\newpage




\newgeometry{left=0.5cm,right=0.5cm,top=3cm,bottom=3cm} 
\begin{landscape}
	\vspace*{\fill}
	\begin{figure}[H]\centering\caption{Important events surrounding FFF and perception in (social) media}

		
		
	
	
	
	
	\begin{subfigure}[h]{0.85\linewidth}\centering\caption{Print media}
		\includegraphics[width=\linewidth]{descriptive/greta_cons_background_timeline_elections_events.pdf}
	\end{subfigure}
	
	
	\par\bigskip\smallskip % force a bit of vertical whitespace
		\begin{subfigure}[h]{0.43\linewidth}\centering\caption{Print media}
			\includegraphics[width=\linewidth]{descriptive/greta_cons_fff_print_media_articles_2019.pdf}
		\end{subfigure}
		\begin{subfigure}[h]{0.43\linewidth}\centering\caption{Twitter feed of Greta Thunberg}
			\includegraphics[width=\linewidth]{descriptive/greta_cons_twitter_gt_favorites_retweets_2019.pdf}
		\end{subfigure}
		
		\begin{minipage}{0.86\linewidth}
			\scriptsize{\emph{Notes:} Panel a presents a timeline of relevant events over the year 2019. Above the timeline, the scheme lists elections that fall in the sample period (election for the European Parliament, Brandenburg, Saxony, and Thuringia). Below the timeline, there is a series of important events, either far-reaching strikes or critical milestones for Greta Thunberg. Panel b shows the daily share of articles that cover FFF (per thousand articles). An article is defined to report about FFF if it contains at least the keyword `Fridays for Future' (in various notations) or `climate strike'. The blue line stems from a moving average with a window of three days. Panel c plots the weekly number of favorites and retweets of Greta Thunberg's tweets (in thousand).\newline \emph{Source:} Own representation with data from Genios and Twitter.}
		\end{minipage}
	\end{figure}
	\vspace*{\fill}\clearpage
\end{landscape}
\restoregeometry








Appendix gives further insights of the content of Greta Thunberg's tweets
Figure \ref{fig_greta_cons:twitter_greta_thunberg_top_words_hashtags_mentions}: top words, hashtags, and mentions
Figure \ref{fig_greta_cons:twitter_greta_sentiment_length}: sentiment and length of tweets


Twitter feed of other German FFF icons: Appendix Figure \ref{fig_greta_cons:twitter_favorites_activists}
















%--------------------------------------------------------------------
% DATA & VARIABLES
%--------------------------------------------------------------------
\newpage
\section{Data}\label{sec_greta_cons:data} 

year 2019, 401 districts (\textit{Kreise})



\subsection{Mobile-Phone Based Tracking Data}

The mobile-phone based tracking data is obtained from \textit{Teralytics}, which draws on the universe of customers from the \textit{Telefonica O$_2$}-network (market share of 31\% in 2019).\footnote{source: https://de.statista.com/statistik/daten/studie/3028/umfrage/marktanteile-der-netzbetreiber-am-mobilfunkmarkt-in-deutschland-seit-1998/} Teralytics uses machine learning-based technology to transform mobile signals into number of journeys between two locations. For the year of 2019, the data contains 64.4 billion trips. There is a required minimum time between the movements so that they count as a journey and it depends on the mode of transport.\footnote{The required minimum time is 30 minutes by car, 60 minutes by train, and 120 minutes by plane.} 


The data reports the daily number of trips between origin-destination pairs.\footnote{Due to data privacy regulations, Teralytix is not allowed to report journeys if there are less than five trips, given the origin-destination pair. We set the missing counts to zero.} The spatial level of aggregation mostly coincides with district borders, only for larger metropolitan areas, there are regions with smaller clusters. For this reason, the data set provides us the number of journeys for 513 different regions, which result in more than 260,000 possible origin-destination pairs. If the length of journey exceeds 30 km, the data can distinguish between different modes of transport: car (6.4 percent of all trips), train (0.9 percent), and plane (< 0.03 percent). The vast majority of trips exhibit shorter distances or cannot be attributed to on of the previous transportation modes and fall into the category "not classified" (92.7 percent).






\begin{itemize}
	\item provider is typical carrier for young people - with the help of regional market shares, there is extrapolation to entire population
\end{itemize}














\subsection{Strike Data}

% map: all strikes 2019
\begin{figure}[H]\centering
	\caption{Strikes in 2019}\label{fig_greta_cons:fff_strikes_2019}
	\includegraphics[width=0.8\linewidth]{descriptive/greta_cons_fff_strikes_2019.png}
	\begin{minipage}{0.8\linewidth}
		\scriptsize{\emph{Notes:} }
	\end{minipage}
\end{figure}

variation across months of the year in Figure \ref{fig_greta_cons:fff_strikes_months} in the Appendix


3,943 strikes (1,966 authorities, 1,592 fff website and web-archive, 385 social media- filling up information for largest cities that were missing in the database)

% temporal variation of strike number across sources
\begin{figure}[H]\centering
	\caption{Number of strikes across sources}\label{fig_greta_cons:number_strikes_per_source}
	\includegraphics[width=0.8\linewidth]{descriptive/greta_cons_number_strikes_per_source}
	\begin{minipage}{0.8\linewidth}
		\scriptsize{\emph{Notes:} daily number of strikes}
	\end{minipage}
\end{figure}


\subsection{Weather Data}


\subsection{Holidays}


\subsection{Election Outcomes}


\subsection{Media Outlets}


\subsection{Population Figures and Other Regional Controls}






%--------------------------------------------------------------------
% MEASUREMENT STRIKE PARTICIPATION
%--------------------------------------------------------------------
\newpage
\section{Granular Measurement of Strike Participation}\label{sec_greta_cons:measurement_strike_participation}


% Residualize journeys %--------------------------------------------------------------------
\subsection{Residualize Journeys and Combine with Strike Data Base}
\begin{align}
	\text{journeys}_{ijt} = \alpha + \vartheta_{ij} + \gamma_t + \text{weather}_{ijt} + \text{holiday}_{ijt} + \varepsilon_{ijt} \label{eq_greta_cons:res_journeys}
\end{align}

\begin{itemize}
	\item origin-destination pair $ij$ (county-level)
	\item $\gamma_t$ contains dow, woy, month
	\item interactions between $\vartheta_{ij} \times \gamma_t$: no interaction, partial interaction(interaction w/ woy and month), and fully interacted
	\item OLS (baseline) and Poisson as robustness 
	\item get residual: $e_{ijt} =(\text{journeys}_{ijt} - \widehat{\text{journeys}}_{ijt})$
	\item match $e_{ijt}$ with set of destinations $j\in\{1,...,J\}$ in which a strike takes place on $t\in\{1,...,T\}$ (mostly Fridays)
	\item $e_{ijt}$ indicates how many people are coming from county $i$ to strike in county $j$ happening in $t$
\end{itemize}



% fig: strike participation for selected strikes
\begin{figure}[H]\centering
	\caption{Strike participation for selected strikes}
	\label{fig_greta_cons:strike_participation_hh_ber}
	% Hamburg
	\begin{subfigure}[h]{0.45\linewidth}\centering
		\includegraphics[width=\linewidth]{descriptive/greta_cons_strike_participation_hh_ols.png}
	\end{subfigure}
	%Berlin
	\begin{subfigure}[h]{0.45\linewidth}\centering
		\includegraphics[width=\linewidth]{descriptive/greta_cons_strike_participation_ber_ols.png}
	\end{subfigure}
	%	%Aachen
	%	\begin{subfigure}[h]{0.4\linewidth}\centering
	%		\includegraphics[width=\linewidth]{descriptive/greta_cons_strike_participation_aa_ols.png}
	%	\end{subfigure}
	\begin{minipage}{0.9\linewidth}
		\scriptsize{\emph{Notes:} Residualized movements [in thousand]. thin lines: ids of teralytics}
	\end{minipage}
\end{figure}


for Hamburg, alternative participation measures shown in Figure \ref{fig_greta_cons:strike_participation_hh_different_measure}

validation exercise: where do spectators at soccer games come from -> Figure \ref{fig_greta_cons:participation_soccer_games}






% PATRICIPATION INDEX %--------------------------------------------------------------------
\subsection{Participation Index}
	
\begin{itemize}	
	\item generate participation index for the origin regions
\end{itemize}
\begin{align}
	\text{Participation Index}_{m\tilde{t}} = \left( \frac{w_{m}}{\text{population}_{m}} \sum\limits_{t=1}^{\tilde{t}}\sum\limits_{j=1}^{J} e_{ijt}\right) \label{eq_greta_cons:participation_index}
\end{align}

\begin{itemize}
	\item relative measure of relative strike participation in municipality $m$ at time period $\tilde{t}$
	\item where $\sum_{t=1}^{\tilde{t}}\sum_{j=1}^{J} e_{ijt}$ indicates the cumulative strike participation in county $i$ up to time $\tilde{t}$ (e.g. when an election takes place).\footnote{As strike participation should have a positive value, we only select non-negative residuals.} We sum over destinations $j$ to account for the fact that there might be more than one destination with climate strikes where individuals from $i$ are going to.
	\item the weights $w_{mi}$ allocate the strike participation from the county to the municipality level. They are defined as $w_m=\tfrac{\text{population }[10-34]_m}{\sum_{r=1}^{R}\text{population }[10-34]_r}$, i.e. the share of young people (aged 10-34) living in municipality $m$ compared to other municipalities $r\in\{1,...,R\}$ in county $i$.
	\item finally we take the number of cumulative strike participants in municipality $m$ at time period $\tilde{t}$ and divide it by the municipality's population figure to have a relative measure.
\end{itemize}















%--------------------------------------------------------------------
% ASSOCIATION WITH ELECTION OUTCOMES
%--------------------------------------------------------------------
\clearpage
\section{Strike Participation and Electoral Outcomes [Helmut \& Maria]}\label{sec_greta_cons:strike_participation_elections}
\subsection{Empirical Strategy}

\begin{align}
	\text{Share Greens}_{me} = \theta_s + \tau_e + \beta\cdot\text{Participation Index}_{me} + \mu \mathbf{X}_{1m} + \xi_{me}\label{eq_greta_cons:vote_share_greens}
\end{align}
\begin{itemize}
	\item $\theta_s$ state fixed-effects, $\tau_e$ election fixed-effects, $\text{Participation Index}_{me}$ participation index up to time when election $e$ takes place, $\mathbf{X}_{1m}$ municipality-levels controls 
	\item all variables standardized
\end{itemize}

\begin{align}
	\Delta(\text{Share Greens}_{me,2019-2015}) = \theta_s + \tau_e + \beta\cdot\text{Participation Index}_{me} + \mu \mathbf{X}_{2m} + \xi_{me}\label{eq_greta_cons:fd_vote_share_greens}
\end{align}
\begin{itemize}
	\item take first-differences to eliminate time-invariant municipality specific effects (unobserved)
	\item $\mathbf{X}_{2m}$ potentially different to $\mathbf{X}_{1m}$
	\item in both specifications: 
	\begin{itemize}
		\item observations are weighted by the population
		\item standard errors are clustered on county level (maybe we should go a bit higher, as this costs us quite some stars)
	\end{itemize}
	
\end{itemize}



\subsection{Results}

% maps greens, fd_greens, strike_index, correlation of the two
\begin{figure}[H]\centering
	\caption{Spatial correlation of the vote share of the Greens and strike participation}
	\label{fig_greta_cons:spatial_correlation_greens_index}
	% greens
	\begin{subfigure}[h]{0.45\linewidth}\centering
		\includegraphics[width=\linewidth]{descriptive/greta_cons_the_greens_eu_election_2019_ags8.png}
	\end{subfigure}
	% fd_greens
	\begin{subfigure}[h]{0.45\linewidth}\centering
		\includegraphics[width=\linewidth]{descriptive/greta_cons_fd_the_greens_eu_election_2019_ags8.png}
	\end{subfigure}
	\begin{subfigure}[h]{0.45\linewidth}\centering
		\includegraphics[width=\linewidth]{descriptive/greta_cons_particiaption_index_eu_election_2019_ags8.png}
	\end{subfigure}
	\begin{subfigure}[h]{0.45\linewidth}\centering
		\includegraphics[width=\linewidth]{descriptive/bivariate_map_legend}
	\end{subfigure}

	\begin{minipage}{0.9\linewidth}
		\scriptsize{\emph{Notes:} quantiles in Panels A-C}
	\end{minipage}
\end{figure}


mention \cite{cantoni2020persistence}


% Tab - first results - assocation strike part & greens
\begin{table}[H]\centering
	\begin{threeparttable}
		\caption{Strike participation and the greens' vote share}\label{tab_greta_cons:associations_part_greens}
		{\def\sym#1{\ifmmode^{#1}\else\(^{#1}\)\fi} 
			\begin{tabular}{l*{3}{c}}
				\toprule
				&\multicolumn{1}{c}{(1)}&\multicolumn{1}{c}{(2)}&\multicolumn{1}{c}{(3)}\\
				& Greens & $\Delta$ Greens & $\Delta$ Greens \\
				& 2019		 & 2019-2015		& 2019-2015 \\
				\midrule
				Participation Index     &      0.1327\sym{***}&      0.0570\sym{**}	 	&	0.0626\sym{**}	\\
										&    (0.0314)         &    (0.0251)         	&	(0.0313)		\\
				\\	
				Observations        	&      12,189         &      12,189         	&	452				\\
				$R^2$               	&       0.570         &       0.755         	&	0.823			\\
				State FE				& \checkmark 		  & \checkmark       		& \checkmark 		\\
				Election FE				& \checkmark 		  & \checkmark       		& \checkmark 		\\
				\bottomrule
		\end{tabular}}
		\begin{tablenotes} 
			\item \scriptsize \emph{Notes:}  \newline Significance levels: * p < 0.10, ** p < 0.05, *** p < 0.01.
		\end{tablenotes} 
	\end{threeparttable}
\end{table}



% Tab: inclusion controls
\begin{table}[H]\centering
	\begin{threeparttable}
		\caption{Inclusion of controls}\label{tab_greta_cons:inclusion_controls}
		{\def\sym#1{\ifmmode^{#1}\else\(^{#1}\)\fi} 
			\begin{tabular}{l*{4}{c}}
				\toprule
				&\multicolumn{1}{c}{(1)}&\multicolumn{1}{c}{(2)}&\multicolumn{1}{c}{(3)}&\multicolumn{1}{c}{(4)}\\
				& Baseline & Income & Unemployment &  Demographics \\
				\midrule
				Participation Index     &      0.0570\sym{**} &      0.0614\sym{***}&      0.0606\sym{**} &      0.0436\sym{*}  \\
										&    (0.0251)         	&    (0.0209)         &    (0.0254)         &    (0.0239)         \\
				\\
				Observations        	&      12,189         &      12,185         &      12,066         &      12,189         \\
				$R^2$              		&       0.755         &       0.800         &       0.755         &       0.778         \\			
				State FE				& \checkmark 		  & \checkmark       & \checkmark 	& \checkmark \\
				Election FE				& \checkmark 		  & \checkmark       & \checkmark   & \checkmark \\
				\bottomrule
		\end{tabular}}
		\begin{tablenotes} 
			\item \scriptsize \emph{Notes:}  \newline Significance levels: * p < 0.10, ** p < 0.05, *** p < 0.01.
		\end{tablenotes} 
	\end{threeparttable}
\end{table}




% Tab: other participation measures
\begin{table}[H]\centering
	\begin{threeparttable}
		\caption{Alternative participation measures}\label{tab_greta_cons:alternative_participation_measures}
		{\def\sym#1{\ifmmode^{#1}\else\(^{#1}\)\fi} 
			\begin{tabular}{l*{3}{c}}
				\toprule
				&\multicolumn{1}{c}{(1)}&\multicolumn{1}{c}{(2)}&\multicolumn{1}{c}{(3)}\\
				& Baseline & \clb{c}{Partial\\interaction} & \clb{c}{Fully\\interacted} \\
				\midrule\\
				
				% OLS
				\multicolumn{4}{l}{\textbf{\textit{Panel A: OLS}}} \\
				Participation Index     &      0.0570\sym{**} &      0.0586\sym{**} &      0.0497\sym{*}  \\  
										&    (0.0251)         &    (0.0260)         &    (0.0290)         \\  
				Observations      		&      12,189         &      12,189         &      12,189         \\  
				$R^2$             		&       0.755         &       0.755         &       0.754         \\ 
				State FE				& \checkmark 		  & \checkmark       & \checkmark  \\
				Election FE				& \checkmark 		  & \checkmark       & \checkmark  \\
				\\ 
				
				% Poisson
				\multicolumn{4}{l}{\textbf{\textit{Panel B: Poisson}}} \\
				Participation Index     &      0.0460\sym{*}  &      0.0482\sym{**} &      0.0475		  \\
										&    (0.0241)         &    (0.0244)         &    (0.0293)         \\
				Observations      		&      12,189         &      12,189         &      12,189         \\
				$R^2$             		&       0.754         &       0.754         &       0.754         \\		
				State FE				& \checkmark 		  & \checkmark       & \checkmark  \\
				Election FE				& \checkmark 		  & \checkmark       & \checkmark  \\
				\bottomrule
		\end{tabular}}
		\begin{tablenotes} 
			\item \scriptsize \emph{Notes:}  \newline Significance levels: * p < 0.10, ** p < 0.05, *** p < 0.01.
		\end{tablenotes} 
	\end{threeparttable}
\end{table}



% figure cutoff distances
\begin{figure}[H]\centering
	\caption{Cutoff distances for origins}\label{fig_greta_cons:origin_cutoff_distances}
	\includegraphics[width=0.9\linewidth]{analysis/greta_cons_cutoff_distance_election_outcomes}
	\begin{minipage}{0.9\linewidth}
		\scriptsize{\emph{Notes:} }
	\end{minipage}
\end{figure}





%--------------------------------------------------------------------
% CONCLUSION
%-------------------------------------------------------------------
\bigskip
\section{Concluding remarks [Helmut \& Maria]}\label{sec_greta_cons:conclusion}