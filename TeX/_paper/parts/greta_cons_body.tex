literature: 
\begin{itemize}
	\item mobile-phone based tracking data: \cite{dave2020contagion}
\end{itemize}



%--------------------------------------------------------------------
% BACKGROUND
%--------------------------------------------------------------------
\bigskip
\section{Background}\label{sec_greta_cons:background}



\textbf{Timeline (important dates)}
\begin{itemize}
	\item 20.08.2018 First time in front of Swedish Parliament
	\item 27.08.2018 First coverage in German newspaper
	\item 3.12.2018-14.12.2018 Katowice (climate conference )
	\item mid December first climate strikes in Germany
	\item 23.01.2019-25.01.2019 Davos (world economic forum)
	\item 01.03.2019 Greta visits Hamburg for climate strike
	\item 15.03.2019 Worldwide climate strikes, more than 1.4 million people involved
	\item 29.03.2019 Greta visits Berlin for climate strike
	\item 16.04.2019 Strasbourg speech at EU
	\item 24.05.2019 2nd Global Climate Strike (for EU elections)
	\item 21.06.2019 Aachen: Climate Justice w/o borders
	\item 14.08.-28.08.2019 Journey across Atlantic 
	\item 20.09.-27.09.2019 Global Week of Climate Action
	\item 23.09.-29.09.2019 NY - UN climate action summit
	\item 29.11.2019 Fourth Global Climate Strike
	\item 02.12.-13.12.2019 Madrid (UN climate change conference )
	\item 11.12.2019 Greta Thunberg Time Person of the Year	
\end{itemize}
see coverage of Greta Thunberg in print media (daily and weekly outlets) in Figures \ref{fig_greta_cons:genios_greta_per_1000_2019} and \ref{fig_greta_cons:genios_greta_per_1000_events_2019}





% notes 
movement initiated by Greta Thunberg, who initiated them in August 2018
she stands as no-one else as 

school climate strike movement (FFF)
`The Greta Effect' (e.g. The Guardian )Influence on the world stage

prizes 
two consecutive nominations for the Nobel Peace Prize (2019 \& 2020)







\newpage




\newgeometry{left=0.5cm,right=0.5cm,top=3cm,bottom=3cm} 
\begin{landscape}
	\vspace*{\fill}
	\begin{figure}[H]\centering\caption{Important events surrounding FFF and perception in (social) media}

		
		
	
	
	
	
	\begin{subfigure}[h]{0.85\linewidth}\centering\caption{Timeline of events}
		\includegraphics[width=\linewidth]{descriptive/greta_cons_background_timeline_elections_events.pdf}
	\end{subfigure}
	
	
	\par\bigskip\smallskip % force a bit of vertical whitespace
		\begin{subfigure}[h]{0.43\linewidth}\centering\caption{Appearance in print media}
			\includegraphics[width=\linewidth]{descriptive/greta_cons_fff_print_media_articles_2019.pdf}
		\end{subfigure}
		\begin{subfigure}[h]{0.43\linewidth}\centering\caption{Twitter feed of Greta Thunberg}
			\includegraphics[width=\linewidth]{descriptive/greta_cons_twitter_gt_favorites_retweets_2019.pdf}
		\end{subfigure}
		
		\begin{minipage}{0.86\linewidth}
			\scriptsize{\emph{Notes:} Panel a presents a timeline of relevant events over the year 2019. Above the timeline, the scheme lists elections that fall in the sample period (election for the European Parliament, Brandenburg, Saxony, and Thuringia). Below the timeline, there is a series of important events, either far-reaching strikes or critical milestones for Greta Thunberg. Panel b shows the daily share of print articles covering FFF (per thousand articles). An article is defined to report about FFF if it contains at least the keyword `Fridays for Future' (in various notations) or `climate strike'. The blue line stems from a moving average with a window of three days. Panel c plots the weekly number of favorites and retweets of Greta Thunberg's tweets (in thousand).\newline \emph{Source:} Own representation with data from Genios and Twitter.}
		\end{minipage}
	\end{figure}
	\vspace*{\fill}\clearpage
\end{landscape}
\restoregeometry








Appendix gives further insights of the content of Greta Thunberg's tweets
Figure \ref{fig_greta_cons:twitter_greta_thunberg_top_words_hashtags_mentions}: top words, hashtags, and mentions
Figure \ref{fig_greta_cons:twitter_greta_sentiment_length}: sentiment and length of tweets


Twitter feed of other German FFF icons: Appendix Figure \ref{fig_greta_cons:twitter_favorites_activists}
















%--------------------------------------------------------------------
% DATA & VARIABLES
%--------------------------------------------------------------------
\newpage
\section{Data}\label{sec_greta_cons:data} 
The data set for the analysis draws on data from 2019 and contains all German regions, either on the county (\textit{Kreise}, $N=401$) or municipality (\textit{Gemeinde}, $N=10,719$) level. We combine six sources of data to residualize journeys and to investigate associations between climate strike participation and electoral outcomes.




% mobile phone data
\subsection{Mobile Phone-Based Tracking Data}
The mobile phone-based tracking data is obtained from \textit{Teralytics}, which draws on the universe of customers from the \textit{Telefonica O$_2$}-network (market share of 31 percent in 2019).\footnote{source: https://de.statista.com/statistik/daten/studie/3028/umfrage/marktanteile-der-netzbetreiber-am-mobilfunkmarkt-in-deutschland-seit-1998/} The mobile network provider is a popular carrier among young people. To obtain representative mobility patterns, \textit{Teralytics} extrapolates to the entire population with the help of regional market shares. \textit{Teralytics} uses machine learning-based technology to transform mobile signals into number of journeys between two locations. For the year 2019, the data contains 64.4 billion trips. There is a required minimum time between the movements so that they count as a journey and it depends on the mode of transport.\footnote{The required minimum time is 30 minutes by car, 60 minutes by train, and 120 minutes by plane.} 

The data reports the daily number of trips between origin-destination pairs over the year 2019. Due to data privacy regulations, \textit{Teralytics} cannot report journeys if there are less than five trips for a given origin-destination pair. The spatial level of aggregation mostly coincides with county borders, only for larger metropolitan areas, there are regions with smaller clusters. For this reason, the data set provides us the number of journeys for 513 different regions, which result in more than 260,000 possible origin-destination pairs. If the length of the journey exceeds 30 km, the data can distinguish between different modes of transportation: car (6.4 percent of all trips), train (0.9 percent), and plane (< 0.03 percent).\footnote{The distance refers to the linear distance between the centroids of two geographies.} The vast majority of trips exhibit shorter distances or cannot be attributed to one of the previous transportation modes and fall into the category "not classified" (92.7 percent).






% strike data
\subsection{Climate Strike Data}
Our data on climate strikes is self-collected and comes from local authorities, social media, and the website of FFF Germany. All public gatherings such as rallies and demonstrations must be registered in Germany. The registration usually happens at the police, the city council, or other regulatory agencies. The large bulk of our database is retrieved by contacting the respective authorities and requesting a list of all climate strikes that took place in their jurisdictions in 2019. Consequently, we obtained information on the exact location and time of nearly 2,000 strikes. In the next step, we used social media postings (Twitter, Facebook, and Instagram) to fill up missing information for the 20 largest cities that were missing in the database. Searching for information on social media channels gave us almost another 400 climate strikes and ensured that we do not miss any large-scale climate strike. In the last step, we enriched our database with information on almost 1,600 strikes retrieved from the website of FFF Germany. Using the internet archive of \url{http://web.archive.org/}, we could restore snapshots with information on climate strikes for some weeks. The locations where the climate strikes took place are geographically encoded. Figure \ref{fig_greta_cons:fff_strikes_2019} shows a map with all strikes contained in our database. There is quite some variation in the strike behavior across the year as depicted in the maps in Appendix Figure \ref{fig_greta_cons:fff_strikes_months}. The same impression is obtained from Figure \ref{fig_greta_cons:number_strikes_per_source}, which presents the daily number of strikes across the three sources. The highest number of strikes is reported on the global climate strike events (March, May, September, and November). Furthermore, it is visible that the strikes from the FFF website are only available in 19 weeks when the domain was archived. The other sources, however, exhibit more continuity in how they report the number of strikes.\footnote{Because of the extreme volatility of the strikes from the FFF Germany website - no information in most instances vs. detailed regional resolution on others-, we first focus on strikes for which there is more continuous information, i.e. strikes obtained from the authorities and social media.} 


% map: all strikes 2019
\begin{figure}[t]\centering
	\caption{Strikes in 2019}\label{fig_greta_cons:fff_strikes_2019}
	\includegraphics[width=0.8\linewidth]{descriptive/greta_cons_fff_strikes_all_2019_200.png}
	\begin{minipage}{0.8\linewidth}
		\scriptsize{\emph{Notes:} The map shows the climate strikes (red dots) in our data base over the year 2019. The bold white lines indicate state boundaries and the thin white lines represent county boundaries.}
	\end{minipage}
\end{figure}


% temporal variation of strike number across sources
\begin{figure}[t]\centering
	\caption{Number of strikes across sources}\label{fig_greta_cons:number_strikes_per_source}
	\includegraphics[width=0.8\linewidth]{descriptive/greta_cons_number_strikes_per_source_ALL_STRIKES}
	\begin{minipage}{0.8\linewidth}
		\scriptsize{\emph{Notes:} The figure shows the daily number of strikes per source. The indicated dates above the spikes mark the four global climate strikes.}
	\end{minipage}
\end{figure}






% Election Results
\subsection{Electoral Data}
For the European Parliament election, we use data provided by the Federal Statistical Office and the statistical offices of the Länder. For the state elections in Brandenburg, Saxony, and Thuringia, we draw on data from the State Returning Officers (\textit{Landeswahlleiter}) and the statistical offices of the Länder. In our analysis, we use the municipality-level vote share for the Greens as our dependent variable:\footnote{Our analysis excludes municipalities (around 5 percent) that were subject to territorial reforms between the elections 2015-2019.} the number of votes for the Greens divided by the total number of valid votes cast. In addition to the vote shares at the four elections in 2019, we also look at first-differences that use the corresponding counterparts in 2015. In our sample, the average vote share for the Greens is 19.68 percent. Table \ref{tab_greta_cons:data_greens_vote_share} shows that there is quite some heterogeneity across elections.




\begin{table}[ht]\centering
	\begin{threeparttable}
		\caption{Vote share for the Greens across elections in 2019}
		\label{tab_greta_cons:data_greens_vote_share}
		\begin{tabular*}{.7\linewidth}{@{\extracolsep{\fill}}l*{5}{c}}
			\toprule
			%			&\multicolumn{1}{c}{(1)}&\multicolumn{1}{c}{(2)}&\multicolumn{1}{c}{(3)}&\multicolumn{1}{c}{(4)}&\multicolumn{1}{c}{(5)}\\
			
			&\multicolumn{2}{c}{Greens}&\multicolumn{2}{c}{$\Delta$ Greens}\\
			&\multicolumn{2}{c}{(2019)}&\multicolumn{2}{c}{(2019-2015)}\\
			\cmidrule(lr){2-3}\cmidrule(lr){4-5}
			Election		&	mean		&	sd		&	mean	& sd	&	Obs.	\\
			\midrule\\
			
			EU				&	20.762		&	7.537	&	9.731	& 3.859	&	10,719	\\
			Brandenburg		&	10.167		&	5.120	&	4.308	& 2.189	&	413		\\
			Saxony			&	4.937		&	1.906	&	1.305	& 0.905	&	414		\\
			Thuringia		&	5.318		&	3.407	&	-0.572	& 0.968	&	645		\\
			\midrule
			Total			&	19.676		&	8.234	&	9.113	& 4.325	&	12,191	\\
			\bottomrule
		\end{tabular*}
		\begin{tablenotes} 
			\item \scriptsize \emph{Notes:} The Table reports the population-weighted mean and standard deviation for the Green's vote share in 2019 (columns 2 and 3) and the first-differenced vote share for the Greens (columns 4 and 5) across the 2019 elections. The last column reports the number of municipalities per election. 
		\end{tablenotes} 
	\end{threeparttable}
\end{table}







% weather data
\subsection{Weather Data}
The weather data is derived from Germany's National Meteorological Service (\textit{Deutscher Wetterdienst}). To construct the weather controls, we aggregate weather information to the county level using the weighted average (inverse distances) of the values from the individual weather monitors within a certain radius to the county's centroid.\footnote{The maximum radius depends on the weather variable. It is 30 km for precipitation, while it is 50 km for hours of sunshine and maximum air temperature.} We use daily averages of the following weather variables: hours of sunshine, precipitation, and maximum air temperature. 



% holidays
\subsection{Holidays}
To account for differences in the number of journeys between origin-destination pairs on regular and special days, we include state-level controls for public and school holidays. The data on school holidays is derived from `The Standing Conference of the Ministers of Education and Cultural Affairs of the Länder in the Federal Republic of Germany' (\textit{Kultusminister Konferenz}). The data on public holidays is collected from \url{https://www.schulferien.org/deutschland/feiertage/}. Additionally, we include dummy variables for peculiar days (New Year's Eve and Carnival season), which shift the expected number of journeys between origin-destination pairs.







\subsection{Other Regional Variables}
The Federal Statistical Office and the statistical offices of the Länder provide a database (\textit{Regionaldatenbank}) of detailed statistics by different subject areas at very granular spatial levels. We draw on this database for various purposes. First, we extract municipality-level population figures that are used to construct the strike participation index and to weigh the observations when analyzing the associations between strike participation and electoral results. Second, we collect municipality-level information on topics such as per capita income, unemployment rates, and demographic characteristics that are used as controls.










%--------------------------------------------------------------------
% MEASUREMENT STRIKE PARTICIPATION
%--------------------------------------------------------------------
\clearpage
\section{Granular Measurement of Strike Participation}\label{sec_greta_cons:measurement_strike_participation}


% Residualize journeys %--------------------------------------------------------------------
\subsection{Residualize Journeys and Combine with Strike Data Base}
\begin{align}
	\text{journeys}_{ijt} = \alpha + \vartheta_{ij} + \gamma_t + \text{weather}_{ijt} + \text{holiday}_{ijt} + \varepsilon_{ijt} \label{eq_greta_cons:res_journeys}
\end{align}

\begin{itemize}
	\item origin-destination pair $ij$ (county-level)
	\item $\gamma_t$ contains dow, woy, month
	\item interactions between $\vartheta_{ij} \times \gamma_t$: no interaction, partial interaction(interaction w/ woy and month), and fully interacted
	\item OLS (baseline) and Poisson as robustness 
	\item get residual: $e_{ijt} =(\text{journeys}_{ijt} - \widehat{\text{journeys}}_{ijt})$
	\item match $e_{ijt}$ with set of destinations $j\in\{1,...,J\}$ in which a strike takes place on $t\in\{1,...,T\}$ (mostly Fridays)
	\item $e_{ijt}$ indicates how many people are coming from county $i$ to strike in county $j$ happening in $t$
\end{itemize}



% fig: strike participation for selected strikes
\begin{figure}[H]\centering
	\caption{Strike participation for selected strikes}
	\label{fig_greta_cons:strike_participation_hh_ber}
	% Hamburg
	\begin{subfigure}[h]{0.45\linewidth}\centering
		\includegraphics[width=\linewidth]{descriptive/greta_cons_strike_participation_hh_ols.png}
	\end{subfigure}
	%Berlin
	\begin{subfigure}[h]{0.45\linewidth}\centering
		\includegraphics[width=\linewidth]{descriptive/greta_cons_strike_participation_ber_ols.png}
	\end{subfigure}
	%	%Aachen
	%	\begin{subfigure}[h]{0.4\linewidth}\centering
	%		\includegraphics[width=\linewidth]{descriptive/greta_cons_strike_participation_aa_ols.png}
	%	\end{subfigure}
	\begin{minipage}{0.9\linewidth}
		\scriptsize{\emph{Notes:} The maps show residualized movements (in thousand) for two exemplary strikes. The two strikes were both visited by Greta Thunberg and attracted large populations. A darker shade of green indicates that more people were coming to the climate strike conditional on the controls discussed in the text. The color scale classification is obtained by using the Fisher-Jenks natural breaks algorithm. The red dots mark the strikes' location, gray areas indicate missing data (censored), bold gray lines show state boundaries, and thin gray lines represent the regions defined by \textit{Teralytics}.}
	\end{minipage}
\end{figure}


for Hamburg, alternative participation measures shown in Figure \ref{fig_greta_cons:strike_participation_hh_different_measure}

validation exercise: where do spectators at soccer games come from -> Figure \ref{fig_greta_cons:participation_soccer_games}






% PATRICIPATION INDEX %--------------------------------------------------------------------
\subsection{Participation Index}
	
\begin{itemize}	
	\item generate participation index for the origin regions
\end{itemize}
\begin{align}
	\text{Participation Index}_{m\tilde{t}} = \left( \frac{w_{m}}{\text{population}_{m}} \sum\limits_{t=1}^{\tilde{t}}\sum\limits_{j=1}^{J} e_{ijt}\right) \label{eq_greta_cons:participation_index}
\end{align}

\begin{itemize}
	\item relative measure of relative strike participation in municipality $m$ at time period $\tilde{t}$
	\item where $\sum_{t=1}^{\tilde{t}}\sum_{j=1}^{J} e_{ijt}$ indicates the cumulative strike participation in county $i$ up to time $\tilde{t}$ (e.g. when an election takes place).\footnote{As strike participation should have a positive value, we only select non-negative residuals.} We sum over destinations $j$ to account for the fact that there might be more than one destination with climate strikes where individuals from $i$ are going to.
	\item the weights $w_{mi}$ allocate the strike participation from the county to the municipality level. They are defined as $w_m=\tfrac{\text{population }[minors]_m}{\sum_{r=1}^{R}\text{population }[minors]_r}$, i.e. the share of young people (aged 0-17) living in municipality $m$ compared to other municipalities $r\in\{1,...,R\}$ in county $i$. -> mention to focus on individuals not allowed to vote to not introduce mechanic effect on the outcome variable
	\item finally we take the number of cumulative strike participants in municipality $m$ at time period $\tilde{t}$ and divide it by the municipality's population figure to have a relative measure.
\end{itemize}















%--------------------------------------------------------------------
% ASSOCIATION WITH ELECTION OUTCOMES
%--------------------------------------------------------------------
\clearpage
\section{Strike Participation and Electoral Outcomes [Helmut \& Maria]}\label{sec_greta_cons:strike_participation_elections}
\subsection{Empirical Strategy}

\begin{align}
	\text{Share Greens}_{me} = \theta_s + \tau_e + \beta\cdot\text{Participation Index}_{me} + \mu \mathbf{X}_{1m} + \xi_{me}\label{eq_greta_cons:vote_share_greens}
\end{align}
\begin{itemize}
	\item $\theta_s$ state fixed-effects, $\tau_e$ election fixed-effects, $\text{Participation Index}_{me}$ participation index up to time when election $e$ takes place, $\mathbf{X}_{1m}$ municipality-levels controls 
	\item all variables standardized
\end{itemize}

\begin{align}
	\Delta(\text{Share Greens}_{me,2019-2015}) = \theta_s + \tau_e + \beta\cdot\text{Participation Index}_{me} + \mu \mathbf{X}_{2m} + \xi_{me}\label{eq_greta_cons:fd_vote_share_greens}
\end{align}
\begin{itemize}
	\item take first-differences to eliminate time-invariant municipality specific effects (unobserved)
	\item $\mathbf{X}_{2m}$ potentially different to $\mathbf{X}_{1m}$
	\item in both specifications: 
	\begin{itemize}
		\item observations are weighted by the population
		\item standard errors are clustered on county level (maybe we should go a bit higher, as this costs us quite some stars)
	\end{itemize}
	
\end{itemize}



\subsection{Results}

% maps greens, fd_greens, strike_index, correlation of the two
\begin{figure}[H]\centering
	\caption{Spatial correlation of the vote share of the Greens and strike participation}
	\label{fig_greta_cons:spatial_correlation_greens_index}
	% greens
	\begin{subfigure}[h]{0.45\linewidth}\centering
		\includegraphics[width=\linewidth]{descriptive/greta_cons_the_greens_eu_election_2019_ags8_100.png}
	\end{subfigure}
	% fd_greens
	\begin{subfigure}[h]{0.45\linewidth}\centering
		\includegraphics[width=\linewidth]{descriptive/greta_cons_fd_the_greens_eu_election_2019_ags8_100.png}
	\end{subfigure}
	\begin{subfigure}[h]{0.45\linewidth}\centering
		\includegraphics[width=\linewidth]{descriptive/greta_cons_particiaption_index_eu_election_2019_ags8_100.png}
	\end{subfigure}
	\begin{subfigure}[h]{0.45\linewidth}\centering
		\includegraphics[width=\linewidth]{descriptive/greta_cons_corr_particiaption_greens_eu_election_2019_ags8_100.png}
	\end{subfigure}

	\begin{minipage}{0.9\linewidth}
		\scriptsize{\emph{Notes:} The maps show, at the municipality level, (A) the vote share for the Greens in the 2019 election, (B) the first-difference of the vote share for the greens (2019-2015), (C) the strike participation index at the time of the EU election (standardized), and (D) the bivariate distribution of the first-differenced vote share for the greens and the participation index. The color scales in panels A-C correspond to quintiles. To generate the bivariate color scale in panel D, we blend the two univariate scales (in terciles, $\Delta$ Greens in red and participation index in blue) into one. Bold lines indicate state boundaries, thin lines represent municipality borders.}
	\end{minipage}
\end{figure}


mention \cite{cantoni2020persistence}


% Tab - first results - assocation strike part & greens
\begin{table}[H]\centering
	\begin{threeparttable}
		\caption{Strike participation and the greens' vote share}\label{tab_greta_cons:associations_part_greens}
		{\def\sym#1{\ifmmode^{#1}\else\(^{#1}\)\fi} 
			\begin{tabular}{l*{3}{c}}
				\toprule
				&\multicolumn{1}{c}{(1)}&\multicolumn{1}{c}{(2)}&\multicolumn{1}{c}{(3)}\\
				& Greens & $\Delta$ Greens & $\Delta$ Greens \\
				& 2019		 & 2019-2015		& 2019-2015 \\
				\midrule
			  Participation Index [std.]&      0.1306\sym{***}&      0.0537\sym{**}	 	&	0.0799\sym{**}	\\
										&    (0.0341)         &    (0.0270)         	&	(0.0340)		\\
				\\	
				Observations        	&      12,189         &      12,189         	&	455				\\
				Adjusted $R^2$         	&       0.569         &       0.754         	&	0.826			\\
				State FE				& \checkmark 		  & \checkmark       		& \checkmark 		\\
				Election FE				& \checkmark 		  & \checkmark       		& \checkmark 		\\
				\bottomrule
		\end{tabular}}
		\begin{tablenotes} 
			\item \scriptsize \emph{Notes:} The specifications use election results from the 2019 elections for the EU and the federal states of Brandenburg, Saxony, and Thuringia. The dependent variable is defined as the Greens' vote share, i.e. the number of votes relative to total votes cast. The explanatory variable is the participation index at the time of the elections. As all variables (dependent and explanatory) are standardized, population-weighted coefficients show the change in the outcome variable (in standard deviation units) due to a one standard deviation increase of the participation increase. Columns 1 and 2 show results on the municipality level, while column 3 presents county-level results. All regressions include state and election fixed-effects. Standard errors are clustered at the county level (number of clusters$=401$). \newline Significance levels: * p < 0.10, ** p < 0.05, *** p < 0.01.
		\end{tablenotes} 
	\end{threeparttable}
\end{table}



% Tab: inclusion controls
\begin{table}[H]\centering
	\begin{threeparttable}
		\caption{Inclusion of controls}\label{tab_greta_cons:inclusion_controls}
		{\def\sym#1{\ifmmode^{#1}\else\(^{#1}\)\fi} 
			\begin{tabular}{l*{4}{c}}
				\toprule
				&\multicolumn{1}{c}{(1)}&\multicolumn{1}{c}{(2)}&\multicolumn{1}{c}{(3)}&\multicolumn{1}{c}{(4)}\\
				& Baseline & Income & Unemployment &  Demographics \\
				\midrule
			  Participation Index [std.]&      0.0537\sym{**} &      0.0439\sym{**} &      0.0560\sym{**} &      0.0422\sym{*}  \\
										&    (0.0270)         	&    (0.0217)         &    (0.0269)         &    (0.0256)         \\
				\\
				Observations        	&      12,189         &      12,185         &      12,066         &      12,189         \\
				Adjusted $R^2$         	&       0.754         &       0.798         &       0.755         &       0.778         \\			
				State FE				& \checkmark 		  & \checkmark       & \checkmark 	& \checkmark \\
				Election FE				& \checkmark 		  & \checkmark       & \checkmark   & \checkmark \\
				\bottomrule
		\end{tabular}}
		\begin{tablenotes} 
			\item \scriptsize \emph{Notes:} The dependent variable is defined as the standardized change in Greens' vote share from 2015 to 2019. The baseline specification in column 1 corresponds to the specification in column 2 of Table \ref{tab_greta_cons:associations_part_greens}. Each column adds a different set of control variables. Column 2 adds the logarithm of per capita income, column 3 adds the unemployment rate, and column 4 controls for population density (dummy: median split) and the age structure of the population. Clustered errors are reported in parentheses. See Table \ref{tab_greta_cons:associations_part_greens} for additional details.\newline Significance levels: * p < 0.10, ** p < 0.05, *** p < 0.01.
		\end{tablenotes} 
	\end{threeparttable}
\end{table}





% figure cutoff distances
\begin{figure}[H]\centering
	\caption{Cutoff distances for origins}\label{fig_greta_cons:origin_cutoff_distances}
	\settototalheight{\dimen0}{\includegraphics[width=0.9\linewidth]{analysis/greta_cons_cutoff_distance_ols_80}}%
	\includegraphics[width=0.9\linewidth]{analysis/greta_cons_cutoff_distance_ols_80}
	\llap{\raisebox{\dimen0-1cm}{% move next graphics to top right corner
			\includegraphics[height=2.2cm]{analysis/greta_cons_cutoff_distance_ols_all}
	}}
	\begin{minipage}{0.9\linewidth}
		\scriptsize{\emph{Notes:} The figure reports estimates and confidence intervals of Equation \ref{eq_greta_cons:fd_vote_share_greens} (vertical axis) for different cutoff distances ranging from 0 to 80 kilometers (horizontal axis). The cutoff distance refers to the maximum distance until when counties are considered to be potential origins for a given strike. Gray bars represent 95\% confidence intervals, while blue bars represent 90\% confidence intervals, both arising from clustering standard errors at the county level.}
	\end{minipage}
\end{figure}




% Tab: other participation measures
\begin{table}[H]\centering
	\begin{threeparttable}
		\caption{Alternative participation measures}\label{tab_greta_cons:alternative_participation_measures}
		{\def\sym#1{\ifmmode^{#1}\else\(^{#1}\)\fi} 
			\begin{tabular}{l*{3}{c}}
				\toprule
				&\multicolumn{1}{c}{(1)}&\multicolumn{1}{c}{(2)}&\multicolumn{1}{c}{(3)}\\\\
				&\multicolumn{3}{c}{Interaction of $\vartheta_{ij}$ and $\gamma_t$}\\
				\cmidrule{2-4}
				& Baseline & \clb{c}{Partial\\interaction} & \clb{c}{Fully\\interacted} \\
				\midrule\\
				
				% OLS
				\multicolumn{4}{l}{\textbf{\textit{Panel A: OLS}}} \\
			Participation Index [std.]  &      0.0537\sym{**} &      0.0520\sym{*}  &      0.0483\sym{*}  \\  
										&    (0.0270)         &    (0.0272)         &    (0.0292)         \\  
				Observations      		&      12,189         &      12,189         &      12,189         \\  
				Adjusted $R^2$         	&       0.754         &       0.754         &       0.754         \\ 
				State FE				& \checkmark 		  & \checkmark       & \checkmark  \\
				Election FE				& \checkmark 		  & \checkmark       & \checkmark  \\
				\\ 
				
				% Poisson
				\multicolumn{4}{l}{\textbf{\textit{Panel B: Poisson}}} \\
			Participation Index [std.]	&      0.0541\sym{**} &      0.0469\sym{*}  &      0.0501\sym{*}  \\
										&    (0.0251)         &    (0.0257)         &    (0.0300)         \\
				Observations      		&      12,189         &      12,189         &      12,189         \\
				Adjusted $R^2$         	&       0.754         &       0.754         &       0.754         \\		
				State FE				& \checkmark 		  & \checkmark       & \checkmark  \\
				Election FE				& \checkmark 		  & \checkmark       & \checkmark  \\
				\bottomrule
		\end{tabular}}
		\begin{tablenotes} 
			\item \scriptsize \emph{Notes:} The dependent variable is defined as the standardized change in Greens' vote share from 2015 to 2019. The column header indicates how the process of residualizing journeys is varied. Column 1 uses the baseline specification presented in Equation \ref{eq_greta_cons:res_journeys}. Column 2 further includes interactions of $\vartheta_{ij}$$\times$week and $\vartheta_{ij}$$\times$month. Column3 presents a fully interacted version (plus also including $\vartheta_{ij}$$\times$day-of-the-week). Panel A (B) contains estimates when estimating Equation \ref{eq_greta_cons:res_journeys} with OLS (Poisson). See Table \ref{tab_greta_cons:associations_part_greens} for additional details.\newline Significance levels: * p < 0.10, ** p < 0.05, *** p < 0.01.
		\end{tablenotes} 
	\end{threeparttable}
\end{table}


%--------------------------------------------------------------------
% CONCLUSION
%-------------------------------------------------------------------
\bigskip
\section{Concluding remarks [Helmut \& Maria]}\label{sec_greta_cons:conclusion}