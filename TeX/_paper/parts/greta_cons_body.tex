literature: 
\begin{itemize}
	\item mobile-phone based tracking data: \cite{dave2020contagion}
\end{itemize}



%--------------------------------------------------------------------
% BACKGROUND
%--------------------------------------------------------------------
\newpage
\section{Background}\label{sec_greta_cons:background}

% general intro 
\textit{Fridays for Future} (henceforth FFF) is a global social movement consisting to a large extent of school students that claim comprehensive, fast, and efficient measures for climate protection. The students skip schools, mostly on Fridays, to lay emphasis on their key demand: to adhere to the targets set in the 2015 Paris agreement to lower global carbon dioxide (CO$_2$) emissions in order to limit the associated increase in global warming to 1.5 degrees above pre-industrial levels. The scientific community has stood in solidarity with FFF (\textit{Scientists for Future}), declaring the concerns as justified, and expressing support for the demands put forward by FFF \citep{warren2019thousands,hagedorn2019science}. The movement is organized in local chapters that plan weekly school strikes for the climate and regular large-scale coordinated marches \citep{smith2019window}. In 2019, there have been four large-scale global climate strikes in March, May, September, and November. For instance, alone for the third global climate strike in September 2019, FFF organized 6,000 protests in 185 countries, which mobilized 7.6 million participants \citep{demoor2020protest}. The exact dates of the global climate strikes and other relevant events are presented in the timeline in Panel (a) of Figure \ref{fig_greta_cons:timeline_media_awareness}.


%Greta Thunberg
\textbf{The role of Greta Thunberg.---} The movement received its initial impulse from Greta Thunberg, the then 15-year old Swedish teenager, who started her \textit{Skolstrejk för klimatet} (school strike for climate) in front of the Swedish parliament in August 2018. Her actions quickly received media attention around the world. Greta Thunberg soon started to advocate her goals at important international events such as the 2018 Climate Change Conference in Katowice, the World Economic Forum in Davos in January 2019, in front of the European Parliament on April 16, or the United Nations (UN) climate change conference in Madrid in December 2019. She sailed across the Atlantic to deliver her famous speech at the UN Climate Action Summit in New York in September 2019 (`how dare you?'). Her person is tightly related to the movement's success, not only on the world stage but also on a more local, personal level. A `Greta Effect' has been documented in many dimensions: it ranges from increases in carbon offsetting projects  \citep{guardian2019greta_effect} to a reduction in the number of airline passengers due to `flight-shaming' \citep{economist2019greta_effect}. Most importantly, Greta Thunberg acted as a role model for many young people's decision to take part in the climate strikes. During the first global climate strike on March 15, roughly 45 percent of German students stated that Greta Thunberg influenced their decision to participate in the event \citep{sommer2019fridays}. The effect of Greta Thunberg as a role model is more pronounced for girls than for boys. Greta Thunberg's commitment to fighting for climate change mitigation was acknowledged by various prestigious honors and awards such as the Right Livelihood Award (promoted as the `Alternative Nobel Prize'), being the youngest `Time Person of the Year' in 2019, and two consecutive nominations for the Nobel Peace Prize (in 2019 and 2020).
% missing: she was also exposed to a lot of critics


% Germany and 2019 in general
\textbf{The German context.---} In Germany, the first strikes started in mid December 2018 in a handful of cities.\footnote{The following information on the German context stems, except where otherwise noted, from \cite{sommer2019fridays}.} By mid January 2019, there were already 25,000 protesters in 50 places, according to the organizers of FFF. The movement gained momentum in March due to the visits of Greta Thunberg in Hamburg and Berlin on March 1 and 29,\footnote{The two strikes that were vistited by Greta Thunberg in March 2019 attracted large populations. According to the organizers, there were 10,000 students in Hamburg and 25,000 people in Berlin.} and the first global climate strike on March 15. On that day, there were an estimated 300,000 protesters on German streets and almost 1.8 million protesters worldwide. The second global climate strike took place just before the 2019 elections for the European Parliament in order to allow a significant impact on electoral outcomes \citep{smith2019window}. FFF had declared the 2019 EU election as the `climate election' (\#voteforclimate) and green parties obtained historically good results all across Europe. In Germany, there were 218 events (more than in any other country) and different election surveys indicated that environmental issues and the topic of climate change affected voters in the election \citep{Time2019may24}. The first central global climate strike took place in Aachen on June 21. The strike activity culminated in the Global Week of Climate Action in September. During that week on September 20, the day of the third global climate strike, the largest climate strikes in world history were reported \citep{guardian2019weekofaction}. While more than 7.6 million individuals participated in climate strikes globally, the numbers amounted to 1.4 million protesters in 550 cities in Germany (alone in Berlin there were almost 300,000) \citep{demoor2020protest}. The strikes were timed to increase pressure for actions to mitigate climate change ahead the UN Climate Action Summit in New York. On September 20, the German government also announced their 54 billion climate change package, which included the introduction of a price per ton of emitted CO$_2$. While the introduction may be seen as a success for FFF, the initially suggested price of 10 euros per ton CO$_2$ was significantly lower than what was demanded by FFF and several experts \citep{economist2019klimapaket}. The fourth and last global climate strike in 2019 happened on November 29, again just days before the UN Climate Change Conference started in Madrid. The estimated strike participation had declined and about 630,000 individuals joined the strike \citep{zeit20194cgs}. In late autumn there were sporadic reports about disappointment and resignation with respect to the achievements with the climate package. If the wave was about the break, it was surely done by the onset of the Corona pandemic in the spring of 2020 \citep{ZEIT2020}.



 % Teilnehmer
 \textbf{Who are the participants?---} The results from surveys conducted at strikes allow drawing a profile of FFF participants \citep{sommer2019fridays,demoor2020protest}. During the first global climate strike in March 2019, almost three out of four protesters were students (school or university), and 60 percent of participants were female. Roughly 70 percent of the respondents located themselves center-left which is far above the value for the general population. When looking at (parental) educational attainment, many participants come from middle-class households. The share of mothers (fathers) with a university degree was 45.8 (49.4) percent, values that are twice as high compared to the population average. The protesters are on average largely inexperienced with demonstrations (40 percent no experience, 42 percent few demonstrations) and were mobilized through either social media (37\%) or interpersonal channels (31.5\%), such as friends and schoolmates.

 
 
 
 % rolle von medien -> einfluss auf einstellung
 \textbf{The importance of (social) media.---} The media coverage on the topic of FFF was crucial for the movement's success. The first coverage in Germany started already in the late autumn of 2018.
 While the initial coverage was centered around the violation of the compulsory school attendance, it soon reported more on the contents of the demand made by FFF. Panel (b) of Figure \ref{fig_greta_cons:timeline_media_awareness} shows the share of articles that cover FFF in German print media over the year 2019. The data is obtained via webscraping from Genios that provides an archive containing almost 300 daily and weekly outlets in Germany. There are clear spikes in media coverage around each of the global climate strikes in March, May, September, and November. The maximum share of articles is attained during the Global Week of Climate Action: more than 3.2\% of all print articles in Germany were reporting on FFF. The same tendency holds for social media. Panel (c) of Figure \ref{fig_greta_cons:timeline_media_awareness} shows the weekly number of favorites and retweets of Greta Thunberg's tweets. Again, the attention in Greta Thunberg's tweets culminates with the global climate strikes. The Appendix gives further insights into the content of Greta Thunberg's tweets.\footnote{Figure \ref{fig_greta_cons:twitter_greta_thunberg_top_words_hashtags_mentions} gives an overview of the most widely used words, hashtags, and mentions. Figure \ref{fig_greta_cons:twitter_greta_sentiment_length} displays information on the sentiment and the length of the tweets. Figure \ref{fig_greta_cons:twitter_favorites_activists} plots the weekly number of favorites of influential German climate activists, such as Luisa Neubauer.}



% shift in public opinion
\textbf{Public opinion.---} Related to the increase in media coverage is a rapid shift in public opinion on environmental issues \citep{smith2019window}. The Institute for election research (\textit{Forschungsgruppe Wahlen e.V.}) has been conducting the survey \cite{politbarometer2019} for more than 40 years, which polls regularly attitudes with regard to political parties, current issues, and delivers projections for upcoming elections. One question in the survey asks about respondent's opinion on the two most pressing current issues in the Federal Republic of Germany. The large graph in Panel C of Figure \ref{fig_greta_cons:timeline_media_awareness} shows that the share of respondents who report that the environment as a high-priority issue is increasing significantly from around 10 percent to almost 60\% over the year 2019. The small graph plots the long-run development since 2000 (the year 2019 is highlighted in green). Before the year 2019, an average share of four percent reported that the environment is among the top two pressing issues.\footnote{Before 2019, a maximum share of 14.5 percent of respondents reported the environment as a top issue.} Respondents also reported that they expect the climate strikes and the associated sudden spike in environmental awareness would lead to political changes \citep{ForschungsgruppeWahlen2019FFF}. While 37 percent of respondents expected an impact on politics in April 2019, the share was already 51 percent at the end of June.




 










%In Germany, the idea to rise awareness for ecological issues and fight environmental pollution is not new. There has been a decades-long environmental movement \citep{smith2019window}. Recently, however, school strikes set the tone for environmental protest. 










\newpage





\newgeometry{left=0.5cm,right=0.5cm,top=3cm,bottom=3cm} 
\begin{landscape}
	\vspace*{\fill}
	\begin{figure}[H]\centering\caption{Important events surrounding FFF, perception in (social) media, and environmental awareness in 2019}\label{fig_greta_cons:timeline_media_awareness}
		\begin{subfigure}[h]{0.99\linewidth}\centering\caption{Timeline of events}
			\includegraphics[width=\linewidth]{descriptive/greta_cons_background_timeline_elections_events.pdf}
		\end{subfigure}
		
		\par\bigskip\smallskip % force a bit of vertical whitespace
		\begin{subfigure}[h]{0.33\linewidth}\centering\caption{Appearance in print media}
			\includegraphics[width=\linewidth]{descriptive/greta_cons_fff_print_media_articles_2019.pdf}
		\end{subfigure}
		\begin{subfigure}[h]{0.33\linewidth}\centering\caption{Twitter feed of Greta Thunberg}
			\includegraphics[width=\linewidth]{descriptive/greta_cons_twitter_gt_favorites_retweets_2019.pdf}
		\end{subfigure}
		\begin{subfigure}[h]{0.33\linewidth}\centering\caption{Environmental awareness}
			\includegraphics[width=\linewidth]{descriptive/greta_cons_gesis_env_issue_combined}
		\end{subfigure}

		\begin{minipage}{\linewidth}
			\scriptsize{\emph{Notes:} Panel a presents a timeline of relevant events over the year 2019. Above the timeline, the scheme lists elections that fall in the sample period (election for the European Parliament, Brandenburg, Saxony, and Thuringia). Below the timeline, there is a series of important events, either far-reaching strikes or critical milestones for Greta Thunberg. Panel b shows the daily share of print articles covering FFF (per thousand articles). An article is defined to report about FFF if it contains at least the keyword `Fridays for Future' (in various notations) or `climate strike'. The blue line stems from a moving average with a window of three days. Panel c plots the weekly number of favorites and retweets of Greta Thunberg's tweets (in thousand). Panel d presents the share of respondents who state that the environment is the most important issue in the representative election survey \cite{politbarometer2019}. Since 1977, the survey informs about the political mood, projections, and attitudes towards current topics.\newline \emph{Source:} Own representation with data from Genios, Twitter, and Forschungsgruppe Wahlen.}
		\end{minipage}
	\end{figure}
	\vspace*{\fill}\clearpage
\end{landscape}
\restoregeometry


















%--------------------------------------------------------------------
% DATA & VARIABLES
%--------------------------------------------------------------------
\newpage
\section{Data}\label{sec_greta_cons:data} 
The data set for the analysis draws on data from 2019 and contains all German regions, either on the county (\textit{Kreise}, $N=401$) or municipality (\textit{Gemeinde}, $N=10,719$) level. We combine six sources of data to residualize journeys and to investigate associations between climate strike participation and electoral outcomes.




% mobile phone data
\subsection{Mobile Phone-Based Tracking Data}
The mobile phone-based tracking data is obtained from \textit{Teralytics}, which draws on the universe of customers from the \textit{Telefonica O$_2$}-network (market share of 31 percent in 2019).\footnote{source: https://de.statista.com/statistik/daten/studie/3028/umfrage/marktanteile-der-netzbetreiber-am-mobilfunkmarkt-in-deutschland-seit-1998/} The mobile network provider is a popular carrier among young people. To obtain representative mobility patterns, \textit{Teralytics} extrapolates to the entire population with the help of regional market shares. \textit{Teralytics} uses machine learning-based technology to transform mobile signals into number of journeys between two locations. For the year 2019, the data contains 64.4 billion trips. There is a required minimum time between the movements so that they count as a journey and it depends on the mode of transport.\footnote{The required minimum time is 30 minutes by car, 60 minutes by train, and 120 minutes by plane.} 

The data reports the daily number of trips between origin-destination pairs over the year 2019. Due to data privacy regulations, \textit{Teralytics} cannot report journeys if there are less than five trips for a given origin-destination pair. The spatial level of aggregation mostly coincides with county borders, only for larger metropolitan areas, there are regions with smaller clusters. For this reason, the data set provides us the number of journeys for 513 different regions, which result in more than 260,000 possible origin-destination pairs. If the length of the journey exceeds 30 km, the data can distinguish between different modes of transportation: car (6.4 percent of all trips), train (0.9 percent), and plane (< 0.03 percent).\footnote{The distance refers to the linear distance between the centroids of two geographies.} The vast majority of trips exhibit shorter distances or cannot be attributed to one of the previous transportation modes and fall into the category "not classified" (92.7 percent).






% strike data
\subsection{Climate Strike Data}
Our data on climate strikes is self-collected and comes from local authorities, social media, and the website of FFF Germany. All public gatherings such as rallies and demonstrations must be registered in Germany. The registration usually happens at the police, the city council, or other regulatory agencies. The large bulk of our database is retrieved by contacting the respective authorities and requesting a list of all climate strikes that took place in their jurisdictions in 2019. Consequently, we obtained information on the exact location and time of nearly 2,000 strikes. In the next step, we used social media postings (Twitter, Facebook, and Instagram) to fill up missing information for the 20 largest cities that were missing in the database. Searching for information on social media channels gave us almost another 400 climate strikes and ensured that we do not miss any large-scale climate strike. In the last step, we enriched our database with information on almost 1,600 strikes retrieved from the website of FFF Germany. Using the internet archive of \url{http://web.archive.org/}, we could restore snapshots with information on climate strikes for some weeks. The locations where the climate strikes took place are geographically encoded. Figure \ref{fig_greta_cons:fff_strikes_2019} shows a map with all strikes contained in our database. There is quite some variation in the strike behavior across the year as depicted in the maps in Appendix Figure \ref{fig_greta_cons:fff_strikes_months}. The same impression is obtained from Figure \ref{fig_greta_cons:number_strikes_per_source}, which presents the daily number of strikes across the three sources. The highest number of strikes is reported on the global climate strike events (March, May, September, and November). Furthermore, it is visible that the strikes from the FFF website are only available in 19 weeks when the domain was archived. The other sources, however, exhibit more continuity in how they report the number of strikes.\footnote{Because of the extreme volatility of the strikes from the FFF Germany website - no information in most instances vs. detailed regional resolution on others-, we first focus on strikes for which there is more continuous information, i.e. strikes obtained from the authorities and social media.} 


% map: all strikes 2019
\begin{figure}[t]\centering
	\caption{Strikes in 2019}\label{fig_greta_cons:fff_strikes_2019}
	\includegraphics[width=0.8\linewidth]{descriptive/greta_cons_fff_strikes_all_2019_200.png}
	\begin{minipage}{0.8\linewidth}
		\scriptsize{\emph{Notes:} The map shows the climate strikes (red dots) in our data base over the year 2019. The bold white lines indicate state boundaries and the thin white lines represent county boundaries.}
	\end{minipage}
\end{figure}


% temporal variation of strike number across sources
\begin{figure}[t]\centering
	\caption{Number of strikes across sources}\label{fig_greta_cons:number_strikes_per_source}
	\includegraphics[width=0.8\linewidth]{descriptive/greta_cons_number_strikes_per_source_ALL_STRIKES}
	\begin{minipage}{0.8\linewidth}
		\scriptsize{\emph{Notes:} The figure shows the daily number of strikes per source. The indicated dates above the spikes mark the four global climate strikes.}
	\end{minipage}
\end{figure}






% Election Results
\subsection{Electoral Data}
For the European Parliament election, we use data provided by the Federal Statistical Office and the statistical offices of the Länder. For the state elections in Brandenburg, Saxony, and Thuringia, we draw on data from the State Returning Officers (\textit{Landeswahlleiter}) and the statistical offices of the Länder. In our analysis, we use the municipality-level vote share for the Greens as our dependent variable:\footnote{Our analysis excludes municipalities (around 5 percent) that were subject to territorial reforms between the elections 2015-2019.} the number of votes for the Greens divided by the total number of valid votes cast. In addition to the vote shares at the four elections in 2019, we also look at first-differences that use the corresponding counterparts in 2015. In our sample, the average vote share for the Greens is 19.68 percent. Table \ref{tab_greta_cons:data_greens_vote_share} shows that there is quite some heterogeneity across elections.




\begin{table}[ht]\centering
	\begin{threeparttable}
		\caption{Vote share for the Greens across elections in 2019}
		\label{tab_greta_cons:data_greens_vote_share}
		\begin{tabular*}{.7\linewidth}{@{\extracolsep{\fill}}l*{5}{c}}
			\toprule
			%			&\multicolumn{1}{c}{(1)}&\multicolumn{1}{c}{(2)}&\multicolumn{1}{c}{(3)}&\multicolumn{1}{c}{(4)}&\multicolumn{1}{c}{(5)}\\
			
			&\multicolumn{2}{c}{Greens}&\multicolumn{2}{c}{$\Delta$ Greens}\\
			&\multicolumn{2}{c}{(2019)}&\multicolumn{2}{c}{(2019-2015)}\\
			\cmidrule(lr){2-3}\cmidrule(lr){4-5}
			Election		&	mean		&	sd		&	mean	& sd	&	Obs.	\\
			\midrule\\
			
			EU				&	20.762		&	7.537	&	9.731	& 3.859	&	10,719	\\
			Brandenburg		&	10.167		&	5.120	&	4.308	& 2.189	&	413		\\
			Saxony			&	4.937		&	1.906	&	1.305	& 0.905	&	414		\\
			Thuringia		&	5.318		&	3.407	&	-0.572	& 0.968	&	645		\\
			\midrule
			Total			&	19.676		&	8.234	&	9.113	& 4.325	&	12,191	\\
			\bottomrule
		\end{tabular*}
		\begin{tablenotes} 
			\item \scriptsize \emph{Notes:} The Table reports the population-weighted mean and standard deviation for the Green's vote share in 2019 (columns 2 and 3) and the first-differenced vote share for the Greens (columns 4 and 5) across the 2019 elections. The last column reports the number of municipalities per election. 
		\end{tablenotes} 
	\end{threeparttable}
\end{table}







% weather data
\subsection{Weather Data}
The weather data is derived from Germany's National Meteorological Service (\textit{Deutscher Wetterdienst}). To construct the weather controls, we aggregate weather information to the county level using the weighted average (inverse distances) of the values from the individual weather monitors within a certain radius to the county's centroid.\footnote{The maximum radius depends on the weather variable. It is 30 km for precipitation, while it is 50 km for hours of sunshine and maximum air temperature.} We use daily averages of the following weather variables: hours of sunshine, precipitation, and maximum air temperature. 



% holidays
\subsection{Holidays}
To account for differences in the number of journeys between origin-destination pairs on regular and special days, we include state-level controls for public and school holidays. The data on school holidays is derived from `The Standing Conference of the Ministers of Education and Cultural Affairs of the Länder in the Federal Republic of Germany' (\textit{Kultusminister Konferenz}). The data on public holidays is collected from \url{https://www.schulferien.org/deutschland/feiertage/}. Additionally, we include dummy variables for peculiar days (New Year's Eve and Carnival season), which shift the expected number of journeys between origin-destination pairs.







\subsection{Other Regional Variables}
The Federal Statistical Office and the statistical offices of the Länder provide a database (\textit{Regionaldatenbank}) of detailed statistics by different subject areas at very granular spatial levels. We draw on this database for various purposes. First, we extract municipality-level population figures that are used to construct the strike participation index and to weigh the observations when analyzing the associations between strike participation and electoral results. Second, we collect municipality-level information on topics such as per capita income, unemployment rates, and demographic characteristics that are used as controls.










%--------------------------------------------------------------------
% MEASUREMENT STRIKE PARTICIPATION
%--------------------------------------------------------------------
\newpage
\section{Granular Measurement of Strike Participation}\label{sec_greta_cons:measurement_strike_participation}


% Residualize journeys %--------------------------------------------------------------------
\subsection{Residualize Journeys and Combine with Strike Data Base}

%--------------------------------------------
% fig: strike participation for selected strikes
\begin{figure}[t]\centering
	\caption{Strike participation for selected strikes}
	\label{fig_greta_cons:strike_participation_hh_ber}
	% Hamburg
	\begin{subfigure}[h]{0.45\linewidth}\centering
		\includegraphics[width=\linewidth]{descriptive/greta_cons_strike_participation_hh_ols.png}
	\end{subfigure}
	%Berlin
	\begin{subfigure}[h]{0.45\linewidth}\centering
		\includegraphics[width=\linewidth]{descriptive/greta_cons_strike_participation_ber_ols.png}
	\end{subfigure}
	%	%Aachen
	%	\begin{subfigure}[h]{0.4\linewidth}\centering
	%		\includegraphics[width=\linewidth]{descriptive/greta_cons_strike_participation_aa_ols.png}
	%	\end{subfigure}
	\begin{minipage}{0.9\linewidth}
		\scriptsize{\emph{Notes:} The maps show residualized movements (in thousand) for two exemplary strikes. The two strikes were both visited by Greta Thunberg and attracted large populations. A darker shade of green indicates that more people were coming to the climate strike conditional on the controls discussed in the text. The color scale classification is obtained by using the Fisher-Jenks natural breaks algorithm. The red dots mark the strikes' location, gray areas indicate missing data (censored), bold gray lines show state boundaries, and thin gray lines represent the regions defined by \textit{Teralytics}.}
	\end{minipage}
\end{figure}
%--------------------------------------------

In the first part of the empirical analysis, we exploit the mobile phone-based tracking data to measure local strike participation. To determine how many participants attend a climate strike and where they come from, we residualize the number of movements between origins and destinations. We exploit a rich fixed effects model to remove the predictable variation in the number of journeys based on the origin-destination pair, date characteristics, holiday fixed effects, and weather controls. Let $\text{journeys}_{ijt}$ denote the number of journeys between origin $i$ and destination $j$ at time $t$, which in our baseline model is given by 
\begin{align}
\text{journeys}_{ijt} = \alpha + \vartheta_{ij} + \underbrace{\varphi_d+\eta_w+\psi_m}_{\gamma_t} + \lambda \mathbf{X}_{ijt}  + \varepsilon_{ijt}. \label{eq_greta_cons:res_journeys}
\end{align}
Origin-destination fixed effects are captured by $\vartheta_{ij}$ and account for time-invariant mobility patterns between origin-destination pairs (mostly on the county level with few subclusters). They ensure we use within instead of between origin-destination pair variation over time. The time fixed effects vector $\gamma_t$ contains binary variables for the day-of-the-week ($\varphi_d$), the week-of-the-year ($\eta_w$), and the month ($\psi_m$). It allows to flexibly control for weekly and seasonal mobility patterns. The vector $\mathbf{X}_{ijt} $ includes public as well as school holidays and weather controls, both at the origin and destination geography.\footnote{State-level public holiday controls contain dummy variables for All Saints' Day, Ascension Day, Assumption Day, Christmas, Corpus Christi, Epiphany, Easter, German Unity Day, Good Friday, Labor Day, New Year's Day, Penance Day, Pentecost, and Reformation Day. Furthermore, it contains binary variables for Carnival season and News Year's Eve.\newline County-level weather controls include air temperature, hours of sunshine, and precipitation.} The model explains a high proportion of the variance in the number of origin-destination journeys as the adjusted $R^2=0.9696$.


% take residuals and combine with strike database
To obtain the number and origin of people participating in climate strikes, we calculate the difference between the observed number of journeys and the predicted number of journeys conditional on the variables described above $e_{ijt} =(\text{journeys}_{ijt} - \widehat{\text{journeys}}_{ijt})$. The residuals $e_{ijt}$ capture the excess mobility net of the variation stemming from the baseline covariates. We match the daily origin-destination-level residuals with our strike database in the next step. The climate strike database contains the precise time and location for the set of destinations $j\in\{1,...,J\}$ in which a strike takes place on $t\in\{1,...,T\}$ (mostly Fridays). Consequently, $e_{ijt}$ indicates how many people are coming from origin $i$ to a strike in destination $j$ happening on day $t$. 


% example Hamburg and Berlin
Figure \ref{fig_greta_cons:strike_participation_hh_ber} shows intuitively how the resulting strike participation measure looks like for two exemplary strikes in Hamburg and Berlin. Both of the climate strikes were visited by Greta Thunberg and attracted large populations. A darker shade of green indicates that more people in that geography were coming to one of the two strikes (locations marked in red) net of the variation stemming from the controls discussed above. The color scale classification is obtained by applying the Fisher-Jenks natural breaks algorithm. The algorithm searches for natural breaks in the data by minimizing the variance within clusters and maximizing the variance between clusters \citep{jenks1967data}. In both instances, strike participation is a relatively local phenomenon. The highest excess population flows are coming from adjacent geographies.





% robustness - alternative measurements
We perform several sensitivity tests to assess the robustness of our local strike participation measure. Figure \ref{fig_greta_cons:strike_participation_hh_different_measure} shows maps for the climate strike in Hamburg using alternative participation measures.\footnote{The corresponding maps for the climate strike in Berlin can be found in Appendix Figure \ref{fig_greta_cons:strike_participation_ber_different_measure}.} Throughout the paper, we focus on the baseline specification but present results when using the following alternative strike participation measures. First, we expand the baseline specification by including interactions of $\vartheta_{ij}\times\gamma_t$ in Equation \ref{eq_greta_cons:res_journeys}, i.e. origin-destination-by-day-of-the-week fixed effects, origin-destination-by-week-of-the-year fixed effects, and origin-destination-by-month fixed effects. The interactions have the advantage of allowing for systematic changes in the origin-destination-level mobility patterns over the year. The top row maps illustrate different versions of included interactions when estimating the equation with OLS. Panel A displays the baseline specification as already shown in Figure \ref{fig_greta_cons:strike_participation_hh_ber}. Panel B adds interactions of the origin-destination fixed effects with week and month fixed effects. We refrain from also interacting the origin-destination fixed effects with the day-of-the-week fixed effects because the average strike participation would likely be absorbed by the interacted fixed effects in regions where climate strikes happen every Friday. For completeness, we show a fully interacted version in Panel C. Second, we test the robustness of the strike participation measure by estimating Equation \ref{eq_greta_cons:res_journeys} with a Poisson model to account for the discrete nature of the dependent variable. We do so by also showing the different interaction specifications discussed in the OLS case. Throughout all specifications emerges a picture of remarkable consistency. The models identify roughly the same set of regions where strike participants come from, irrespective of the modeling choices. The levels might be different, but this will not be an issue as we use standardized versions of the strike participation measure.



% validation exercise soccer
As a further check, we validate our approach to identify the origin of people attending large-scale public events by investigating the catchment areas for soccer games. In that respect, Figure \ref{fig_greta_cons:participation_soccer_games} presents results for matches happening in Munich, Dortmund, and Freiburg (an example of a large, medium, and small city). Reassuringly, the maps look quite similar to the ones depicting climate strikes and illustrate that the catchment areas lie around the matches' location. However, in comparison to the climate strikes, the catchment areas are more extensive and contain the guest teams' origin.







%--------------------------------------------
% analysis strike particpation robustness: alternative participation measures
\afterpage{
	\newgeometry{left=0.5cm,right=0.5cm,top=3cm,bottom=3cm} 
	\begin{landscape}
		\vspace*{\fill}
		\begin{figure}[H]\centering
			\caption{Alternative participation measures for the climate strike in Hamburg}
			\label{fig_greta_cons:strike_participation_hh_different_measure}
			% Hamburg
			\begin{subfigure}[h]{0.22\linewidth}\centering
				\includegraphics[width=\linewidth]{descriptive/maps_resid_trips/greta_cons_strike_participation_hh_spec_res_ols}
			\end{subfigure}
			\begin{subfigure}[h]{0.22\linewidth}\centering
				\includegraphics[width=\linewidth]{descriptive/maps_resid_trips/greta_cons_strike_participation_hh_spec_res_ols_int_small}
			\end{subfigure}
			\begin{subfigure}[h]{0.22\linewidth}\centering
				\includegraphics[width=\linewidth]{descriptive/maps_resid_trips/greta_cons_strike_participation_hh_spec_res_ols_int_only}
			\end{subfigure}
			
			\begin{subfigure}[h]{0.22\linewidth}\centering
				\includegraphics[width=\linewidth]{descriptive/maps_resid_trips/greta_cons_strike_participation_hh_spec_res_p}
			\end{subfigure}
			\begin{subfigure}[h]{0.22\linewidth}\centering
				\includegraphics[width=\linewidth]{descriptive/maps_resid_trips/greta_cons_strike_participation_hh_spec_res_p_int_small}
			\end{subfigure}
			\begin{subfigure}[h]{0.22\linewidth}\centering
				\includegraphics[width=\linewidth]{descriptive/maps_resid_trips/greta_cons_strike_participation_hh_spec_res_p_int_only}
			\end{subfigure}
			\begin{minipage}{0.9\linewidth}
				\scriptsize{\emph{Notes:} The maps show residualized movements (in thousand) for the climate strike in Hamburg (March, 1). Column 1 uses the baseline specification in Equation \ref{eq_greta_cons:res_journeys}, column 2 further adds $\vartheta_{ij}$$\times\eta_w$ and  $\vartheta_{ij}$$\times\psi_m$, and column 3 presents a fully interacted version (plus also including $\vartheta_{ij}$$\times\varphi_d$). The top row uses OLS to estimate Equation \ref{eq_greta_cons:res_journeys}, while the bottom row shows results when using Poisson. A darker shade of green indicates that more people were coming to the climate strike conditional on the controls discussed in the text. The color scale classification is obtained by using the Fisher-Jenks natural breaks algorithm. See Figure \ref{fig_greta_cons:strike_participation_hh_ber} for additional details.}
			\end{minipage}
		\end{figure}
		\vspace*{\fill}\clearpage
	\end{landscape}
	\restoregeometry
}
%--------------------------------------------







% PATRICIPATION INDEX %--------------------------------------------------------------------
\subsection{Construction of a Strike Participation Index}

After identifying the origin of strike participants, we want to account for the intensity of local participation rates over time. Undoubtedly, there is a difference between regions with continuous strike involvement and regions with one-time events. In this section, we incorporate these considerations by constructing a cumulative municipality-level measure of strike participation across the origin regions. We define our index as follows:
\begin{align}
	\text{Participation Index}_{m\tilde{t}} = \left( \frac{w_{m}}{\text{population}_{m}} \sum\limits_{t=1}^{\tilde{t}}\sum\limits_{j=1}^{J} e_{ijt}\right), \label{eq_greta_cons:participation_index}
\end{align}
where $\text{Participation Index}_{m\tilde{t}}$ is a cumulative measure of relative strike participation in origin municipality $m$ at time period $\tilde{t}$. It comprises the following elements: $\sum_{t=1}^{\tilde{t}}\sum_{j=1}^{J} e_{ijt}$ indicates the cumulative strike participation in county $i$ up to time $\tilde{t}$ (for instance when an election takes place).\footnote{We only select non-negative residuals to obtain throughout positive values of strike participation.} We sum over destinations $j$ to account for the fact that individuals residing in $i$ are going to climate strikes in more than one destination. Furthermore, we sum over all times $t\in\{1,...,\tilde{t}\}$ in which county $i$ appears as an origin for a strike in any destination $j$. The weights $w_{mi}$ allocate the strike participation from the county to the municipality level. They are defined as $w_m=\tfrac{\text{population }[minors]_m}{\sum_{r=1}^{R}\text{population }[minors]_r}$, in other words the share of young people (aged 0-17) living in municipality $m$ compared to other municipalities $r\in\{1,...,R\}$ in county $i$. The weights resemble the idea that strike participation should be higher in municipalities with many young people relative to municipalities with few young people. We restrain our weights to minors, i.e. the population share that is not allowed to vote, with the purpose to not generate associations with electoral outcomes by construction. Finally, we take the number of cumulative strike participants in municipality $m$ at time $\tilde{t}$ and divide it by the municipality's population figure to have a relative measure.

















%--------------------------------------------------------------------
% ASSOCIATION WITH ELECTION OUTCOMES
%--------------------------------------------------------------------
\clearpage
\section{Strike Participation and Electoral Outcomes [Helmut \& Maria]}\label{sec_greta_cons:strike_participation_elections}



% EMPIRICAL STRATEGY %--------------------------------------------------------------------
\subsection{Empirical Strategy}


Our approach to investigating the relationship between strike participation and electoral results is based on a fixed-effects model, which exploits within-state and within-election variation of local strike participation rates. We estimate the following model:\footnote{For some aspects in this section's empirical approach, we found inspiration in the study of \cite{cantoni2020persistence} who investigate the municipality-level relationship between the support for Germany's Nazi party in 1933 and the likelihood to vote for the `Alternative for Germany', an increasingly right-wing populist party.}
\begin{align}
	\text{Share Greens}_{me} = \theta_s + \tau_e + \beta\cdot\text{Participation Index}_{me} + \mu \mathbf{X}_{1m} + \xi_{me},\label{eq_greta_cons:vote_share_greens}
\end{align}
where $\text{Share Greens}_{me}$ is the Greens' vote share in municipality $m$ for election $e$, which we define as the number of votes for the Greens relative to the total number of valid votes cast. We regress the dependent variable on state and election fixed effects ($\theta_s$ and $\tau_e$). The $\text{Participation Index}_{me}$ captures the cumulative relative strike participation up to the time when election $e$ takes place. For some specifications, we also include a set of municipality-level controls $\mathbf{X}_{1m}$, such as income, unemployment, and demographic characteristics. 


% first-difference
To eliminate unobserved time-invariant municipality specific effects, such as a general tendency to vote for the Greens, we employ a first-differences strategy in which we examine the \textit{change} in the vote share. Let $\Delta(\text{Share Greens}_{me,2019-2015})$ denote the change in the Greens' vote share from 2015 to 2019 in municipality $m$ for election $e$, which is given by
\begin{align}
	\Delta(\text{Share Greens}_{me,2019-2015}) = \theta_s + \tau_e + \beta\cdot\text{Participation Index}_{me} + \mu \mathbf{X}_{2m} + \xi_{me}.\label{eq_greta_cons:fd_vote_share_greens}
\end{align}
Although first-differencing removes time-invariant municipality specific effects, we still include in some specifications a set of controls $\mathbf{X}_{2m}$ (potentially different to $\mathbf{X}_{1m}$ in Equation \ref{eq_greta_cons:vote_share_greens}) to allow for time-varying effects of covariates.


In both Equations \ref{eq_greta_cons:vote_share_greens} and \ref{eq_greta_cons:fd_vote_share_greens}, all variables, dependent and explanatory alike, are standardized. $\beta$ is the parameter of interest and captures the relationship between local strike participation rates and the (change in the) greens' vote share. The parameter reports the change in the outcome variable in standard deviation units due to a one standard deviation increase of the participation index. Observations are weighted by population figures from the Federal Statistical Office. We calculate sandwiched standard error estimates allowing errors to be correlated across municipalities corresponding to the same county---i.e., clustered at the county level.

% robustness: 
%cutoff distances, and different




% RESULTS %--------------------------------------------------------------------
\subsection{Results}

% maps greens, fd_greens, strike_index, correlation of the two
\begin{figure}[H]\centering
	\caption{Spatial correlation of the vote share of the Greens and strike participation}
	\label{fig_greta_cons:spatial_correlation_greens_index}
	% greens
	\begin{subfigure}[h]{0.45\linewidth}\centering
		\includegraphics[width=\linewidth]{descriptive/greta_cons_the_greens_eu_election_2019_ags8_100.png}
	\end{subfigure}
	% fd_greens
	\begin{subfigure}[h]{0.45\linewidth}\centering
		\includegraphics[width=\linewidth]{descriptive/greta_cons_fd_the_greens_eu_election_2019_ags8_100.png}
	\end{subfigure}
	\begin{subfigure}[h]{0.45\linewidth}\centering
		\includegraphics[width=\linewidth]{descriptive/greta_cons_particiaption_index_eu_election_2019_ags8_100.png}
	\end{subfigure}
	\begin{subfigure}[h]{0.45\linewidth}\centering
		\includegraphics[width=\linewidth]{descriptive/greta_cons_corr_particiaption_greens_eu_election_2019_ags8_200.png}
	\end{subfigure}

	\begin{minipage}{0.9\linewidth}
		\scriptsize{\emph{Notes:} The maps show, at the municipality level, (A) the vote share for the Greens in the 2019 election, (B) the first-difference of the vote share for the greens (2019-2015), (C) the strike participation index at the time of the EU election (standardized), and (D) the bivariate distribution of the first-differenced vote share for the greens and the participation index. The color scales in panels A-C correspond to quintiles. To generate the bivariate color scale in panel D, we blend the two univariate scales (in terciles, $\Delta$ Greens in red and participation index in blue) into one. Bold lines indicate state boundaries, thin lines represent municipality borders.}
	\end{minipage}
\end{figure}





% Tab - first results - assocation strike part & greens
\begin{table}[H]\centering
	\begin{threeparttable}
		\caption{Strike participation and the greens' vote share}\label{tab_greta_cons:associations_part_greens}
		{\def\sym#1{\ifmmode^{#1}\else\(^{#1}\)\fi} 
			\begin{tabular*}{.8\linewidth}{@{\extracolsep{\fill}}l*{3}{c}}
				\toprule
				&\multicolumn{1}{c}{(1)}&\multicolumn{1}{c}{(2)}&\multicolumn{1}{c}{(3)}\\
				& Greens & $\Delta$ Greens & $\Delta$ Greens \\
				& 2019		 & 2019-2015		& 2019-2015 \\
				\midrule
			  Participation Index [std.]&      0.1306\sym{***}&      0.0537\sym{**}	 	&	0.0799\sym{**}	\\
										&    (0.0341)         &    (0.0270)         	&	(0.0340)		\\
				\\	
				Observations        	&      12,189         &      12,189         	&	455				\\
				Adjusted $R^2$         	&       0.569         &       0.754         	&	0.826			\\
				State FE				& \checkmark 		  & \checkmark       		& \checkmark 		\\
				Election FE				& \checkmark 		  & \checkmark       		& \checkmark 		\\
				\bottomrule
		\end{tabular*}}
		\begin{tablenotes} 
			\item \scriptsize \emph{Notes:} The specifications use election results from the 2019 elections for the EU and the federal states of Brandenburg, Saxony, and Thuringia. The dependent variable is defined as the Greens' vote share, i.e. the number of votes relative to total votes cast. The explanatory variable is the participation index at the time of the elections. As all variables (dependent and explanatory) are standardized, population-weighted coefficients show the change in the outcome variable (in standard deviation units) due to a one standard deviation increase of the participation index. Columns 1 and 2 show results on the municipality level, while column 3 presents county-level results. All regressions include state and election fixed effects. Standard errors are clustered at the county level (number of clusters $=401$). \newline Significance levels: * p < 0.10, ** p < 0.05, *** p < 0.01.
		\end{tablenotes} 
	\end{threeparttable}
\end{table}



% Tab: inclusion controls
\begin{table}[H]\centering
	\begin{threeparttable}
		\caption{Inclusion of controls}\label{tab_greta_cons:inclusion_controls}
		{\def\sym#1{\ifmmode^{#1}\else\(^{#1}\)\fi} 
			\begin{tabular}{l*{4}{c}}
				\toprule
				&\multicolumn{1}{c}{(1)}&\multicolumn{1}{c}{(2)}&\multicolumn{1}{c}{(3)}&\multicolumn{1}{c}{(4)}\\
				& Baseline & Income & Unemployment &  Demographics \\
				\midrule
			  Participation Index [std.]&      0.0537\sym{**} &      0.0439\sym{**} &      0.0560\sym{**} &      0.0422\sym{*}  \\
										&    (0.0270)         	&    (0.0217)         &    (0.0269)         &    (0.0256)         \\
				\\
				Observations        	&      12,189         &      12,185         &      12,066         &      12,189         \\
				Adjusted $R^2$         	&       0.754         &       0.798         &       0.755         &       0.778         \\			
				State FE				& \checkmark 		  & \checkmark       & \checkmark 	& \checkmark \\
				Election FE				& \checkmark 		  & \checkmark       & \checkmark   & \checkmark \\
				\bottomrule
		\end{tabular}}
		\begin{tablenotes} 
			\item \scriptsize \emph{Notes:} The dependent variable is defined as the standardized change in Greens' vote share from 2015 to 2019. The baseline specification in column 1 corresponds to the specification in column 2 of Table \ref{tab_greta_cons:associations_part_greens}. Each column adds a different set of control variables. Column 2 adds the logarithm of per capita income, column 3 adds the unemployment rate, and column 4 controls for population density (dummy: median split) and the age structure of the population. Clustered errors are reported in parentheses. See Table \ref{tab_greta_cons:associations_part_greens} for additional details.\newline Significance levels: * p < 0.10, ** p < 0.05, *** p < 0.01.
		\end{tablenotes} 
	\end{threeparttable}
\end{table}



% Tab - voter_tunrout
\begin{table}[H]\centering
	\begin{threeparttable}
		\caption{Strike participation and voter turnout}\label{tab_greta_cons:associations_part_turnout}
		{\def\sym#1{\ifmmode^{#1}\else\(^{#1}\)\fi} 
			\begin{tabular*}{.68\linewidth}{@{\extracolsep{\fill}}l*{2}{c}}
				\toprule
				&\multicolumn{1}{c}{(1)}&\multicolumn{1}{c}{(2)}\\
				& Turnout & $\Delta$ Turnout  \\
				& 2019		 & 2019-2015	\\
				\midrule
				Participation Index [std.]	&      0.0140		  &      0.0787\sym{***} 			\\
											&    (0.0325)         &    (0.0181)         				\\
				\\	
				Observations        		&      12,189         &      12,189         			\\
				Adjusted $R^2$         		&       0.270         &       0.499         			\\
				State FE					& \checkmark 		  & \checkmark       			\\
				Election FE					& \checkmark 		  & \checkmark       		\\
				\bottomrule
		\end{tabular*}}
		\begin{tablenotes} 
			\item \scriptsize \emph{Notes:} The specifications use election results from the 2019 elections for the EU and the federal states of Brandenburg, Saxony, and Thuringia. The dependent variable is defined as the voter turnout, i.e. the number of votes relative to total votes cast. See Table \ref{tab_greta_cons:associations_part_greens} for additional details. Standard errors are clustered at the county level (number of clusters $=401$). \newline Significance levels: * p < 0.10, ** p < 0.05, *** p < 0.01.
		\end{tablenotes} 
	\end{threeparttable}
\end{table}







% ROBUSTNESS %--------------------------------------------------------------------
% figure cutoff distances
\begin{figure}[H]\centering
	\caption{Cutoff distances for origins}\label{fig_greta_cons:origin_cutoff_distances}
	\settototalheight{\dimen0}{\includegraphics[width=0.9\linewidth]{analysis/greta_cons_cutoff_distance_ols_80}}%
	\includegraphics[width=0.9\linewidth]{analysis/greta_cons_cutoff_distance_ols_80}
	\llap{\raisebox{\dimen0-1cm}{% move next graphics to top right corner
			\includegraphics[height=2.2cm]{analysis/greta_cons_cutoff_distance_ols_all}
	}}
	\begin{minipage}{0.9\linewidth}
		\scriptsize{\emph{Notes:} The figure reports estimates and confidence intervals of Equation \ref{eq_greta_cons:fd_vote_share_greens} (vertical axis) for different cutoff distances ranging from 0 to 80 kilometers (horizontal axis). The cutoff distance refers to the maximum distance until when counties are considered to be potential origins for a given strike. Gray bars represent 95\% confidence intervals, while blue bars represent 90\% confidence intervals, both arising from clustering standard errors at the county level.}
	\end{minipage}
\end{figure}




% Tab: other participation measures
\begin{table}[H]\centering
	\begin{threeparttable}
		\caption{Alternative participation measures}\label{tab_greta_cons:alternative_participation_measures}
		{\def\sym#1{\ifmmode^{#1}\else\(^{#1}\)\fi} 
			\begin{tabular*}{.72\linewidth}{@{\extracolsep{\fill}}l*{3}{c}}
				\toprule
				&\multicolumn{1}{c}{(1)}&\multicolumn{1}{c}{(2)}&\multicolumn{1}{c}{(3)}\\\\
				&\multicolumn{3}{c}{Interaction of $\vartheta_{ij}$ and $\gamma_t$}\\
				\cmidrule{2-4}
				& Baseline & \clb{c}{Partial\\interaction} & \clb{c}{Fully\\interacted} \\
				\midrule\\
				
				% OLS
				\multicolumn{4}{l}{\textbf{\textit{Panel A: OLS}}} \\
			Participation Index [std.]  &      0.0537\sym{**} &      0.0520\sym{*}  &      0.0483\sym{*}  \\  
										&    (0.0270)         &    (0.0272)         &    (0.0292)         \\  
				Observations      		&      12,189         &      12,189         &      12,189         \\  
				Adjusted $R^2$         	&       0.754         &       0.754         &       0.754         \\ 
				State FE				& \checkmark 		  & \checkmark       & \checkmark  \\
				Election FE				& \checkmark 		  & \checkmark       & \checkmark  \\
				\\ 
				
				% Poisson
				\multicolumn{4}{l}{\textbf{\textit{Panel B: Poisson}}} \\
			Participation Index [std.]	&      0.0541\sym{**} &      0.0469\sym{*}  &      0.0501\sym{*}  \\
										&    (0.0251)         &    (0.0257)         &    (0.0300)         \\
				Observations      		&      12,189         &      12,189         &      12,189         \\
				Adjusted $R^2$         	&       0.754         &       0.754         &       0.754         \\		
				State FE				& \checkmark 		  & \checkmark       & \checkmark  \\
				Election FE				& \checkmark 		  & \checkmark       & \checkmark  \\
				\bottomrule
		\end{tabular*}}
		\begin{tablenotes} 
			\item \scriptsize \emph{Notes:} The dependent variable is defined as the standardized change in Greens' vote share from 2015 to 2019. The column header indicates how the process of residualizing journeys is varied. Column 1 uses the baseline specification presented in Equation \ref{eq_greta_cons:res_journeys}. Column 2 further includes interactions of $\vartheta_{ij}$$\times$week and $\vartheta_{ij}$$\times$month. Column3 presents a fully interacted version (plus also including $\vartheta_{ij}$$\times$day-of-the-week). Panel A (B) contains estimates when estimating Equation \ref{eq_greta_cons:res_journeys} with OLS (Poisson). See Table \ref{tab_greta_cons:associations_part_greens} for additional details.\newline Significance levels: * p < 0.10, ** p < 0.05, *** p < 0.01.
		\end{tablenotes} 
	\end{threeparttable}
\end{table}


%--------------------------------------------------------------------
% CONCLUSION
%-------------------------------------------------------------------
\bigskip
\section{Concluding remarks [Helmut \& Maria]}\label{sec_greta_cons:conclusion}