%--------------------------------------------------------------------
%	DOCUMENT CLASS
%--------------------------------------------------------------------
\documentclass[11pt, a4paper]{article} % type of document (paper, presentation, book,...); scrartcl class with sans serif titles, European layout 
\usepackage{fullpage} % leaves less space at margins of page


%--------------------------------------------------------------------
%	INPUT
%--------------------------------------------------------------------
\usepackage[T1]{fontenc} 	% Use 8-bit encoding that has 256 glyphs
\usepackage[utf8]{inputenc} % Required for including letters with accents, Umlaute,...
\usepackage{float} 			% better control over placement of tables and figures in the text
\usepackage{graphicx} 		% input of graphics
\usepackage{xcolor} 		% advanced color package
\usepackage{url} 			% include (clickable) URLs
\usepackage[breaklinks=true]{hyperref}
\usepackage{pdfpages}		% insert pages of external pdf documents
\setlength{\parskip}{0em}	% vertical spacing for paragraphs
\setlength{\parindent}{0em}	% horizonzal spacing for paragraphs
\usepackage{tikz}
\usepackage{tikzscale}		% helps to adjust tikz pictures to textwidth/linewidth
\usetikzlibrary{decorations.pathreplacing}
\usetikzlibrary{patterns}
\usetikzlibrary{arrows}
\usetikzlibrary{automata,positioning}
\usepackage{eurosym}		% Eurosymbol

% Have sections in TOC, but not in text
\usepackage{xparse}% for easier management of optional arguments
\ExplSyntaxOn
\NewDocumentCommand{\TODO}{msom}
{
	\IfBooleanF{#1}% do nothing if it's starred
	{
		\cs_if_eq:NNT #1 \chapter { \cleardoublepage\mbox{} }
		\refstepcounter{\cs_to_str:N #1}
		\IfNoValueTF{#3}
		{
			\addcontentsline{toc}{\cs_to_str:N #1}{\protect\numberline{\use:c{the\cs_to_str:N #1}}#4}
		}
		{
			\addcontentsline{toc}{\cs_to_str:N #1}{\protect\numberline{\use:c{the\cs_to_str:N #1}}#3}
		}
	}
	\cs_if_eq:NNF #1 \chapter { \mbox{} }% allow page breaks after sections
}
\ExplSyntaxOff

%--------------------------------------------------------------------
%	TABLES, FIGURES, LISTS
%--------------------------------------------------------------------
\usepackage{booktabs} 		% better tables
\usepackage{longtable}		% tables that may be continued on the next page
\usepackage{threeparttable} % add notes below tables
\renewcommand\TPTrlap{}		% add margins on the side of the notes
	\renewcommand\TPTnoteSettings{%
	\setlength\leftmargin{5 pt}%
	\setlength\rightmargin{5 pt}%
}
\usepackage[
center, format=plain,
font=normalsize,
nooneline,
labelfont={bf}
]{caption} 				% change format of captions of tables and graphs 
%USED IN MPHIL: \usepackage[labelfont=bf,labelsep = period, singlelinecheck=off,justification=raggedright]{caption}, other specifications which are nice: labelformat = parens -> number in paranthesis 


%\usepackage{threeparttablex} % for "ThreePartTable" environment, helps to combine threepart and longtable

% Allow line breaks with \\ in column headings of tables
\newcommand{\clb}[3][c]{%
	\begin{tabular}[#1]{@{}#2@{}}#3\end{tabular}}

% allow line breaks with \\ in row titles
\usepackage{multirow}

\newcommand{\rlb}[3][c]{%
\multirow{2}{*}{\begin{tabular}[#1]{@{}#2@{}}#3\end{tabular}}}% optional argument: b = bottom or t= top alignment


\usepackage[singlelinecheck=on]{subcaption}%both together help to have subfigures
\usepackage{wrapfig}				% wrap text around figure


\usepackage{rotating}				% rotating figures & tables
\usepackage{enumerate}				% change appearance of the enumerator
\usepackage{paralist, enumitem}		% better enumerations
\setlist{noitemsep}					% no additional vertical spacing for enurations
%--------------------------------------------------------------------
%	MATH
%--------------------------------------------------------------------
\usepackage{amsmath,amssymb,amsfonts} % more math symbols and commands
\let\vec\mathbf				 % make vector bold, with no arrow and not in italic

%--------------------------------------------------------------------
%	LANGUAGE SPECIFICS
%--------------------------------------------------------------------
\usepackage[american]{babel} % man­ages cul­tur­ally-de­ter­mined ty­po­graph­i­cal (and other) rules, and hy­phen­ation pat­terns
\usepackage{csquotes} % language specific quotations

%--------------------------------------------------------------------
%	BIBLIOGRAPHY & CITATIONS
%--------------------------------------------------------------------
\usepackage{csquotes} % language specific quotations
\usepackage{etex}		% some more Tex functionality
\usepackage[nottoc]{tocbibind} %add bibliography to TOC
\usepackage[authoryear, round, comma]{natbib} %biblatex

%--------------------------------------------------------------------
%	PATHS
%--------------------------------------------------------------------
\makeatletter
\def\input@path{{../../analysis/output/tables/}}	%PATH TO TABLES
%or: \def\input@path{{/path/to/folder/}{/path/to/other/folder/}}
\makeatother
\graphicspath{{../../analysis/output/graphs/}}		% PATH TO GRAPHS

%--------------------------------------------------------------------
%	LAYOUT
%--------------------------------------------------------------------
\usepackage[left=3cm,right=3cm,top=3cm,bottom=3cm]{geometry}
\usepackage{pdflscape} % lscape.sty Produce landscape pages in a (mainly) portrait document.

\definecolor{darkblue}{rgb}{0.0,0.0,0.6}
\newcommand\natalia[1]{\textcolor{orange}{#1}}

% CAPTIAL LETTERS FOR SECTION CAPTIONS
%\usepackage{sectsty}
%\sectionfont{\normalfont\scshape\centering\textbf}
%\renewcommand{\thesection}{\Roman{section}.}
%\renewcommand{\thesubsection}{\Alph{subsection}.}%\thesection\Alph{subsection}.
%\subsectionfont{\itshape}
%\subsubsectionfont{\scshape}
%\newcommand\relphantom[1]{\mathrel{\phantom{#1}}}
%\setlength\topmargin{0.1in} \setlength\headheight{0.1in}
%\setlength\headsep{0in} \setlength\textheight{9.2in}
%\setlength\textwidth{6.3in} \setlength\oddsidemargin{0.1in}
%\setlength\evensidemargin{0.1in}

\hypersetup{
  colorlinks  = true,
  citecolor   = darkblue,
 	linkcolor   = darkblue,
  urlcolor    = darkblue 
} % macht die URLS blau   
     
\usepackage{lettrine}	% First letter capitalized

% have date in month year format (i.e. omit the day in dates)
\usepackage{datetime}
\newdateformat{monthyeardate}{%
  \monthname[\THEMONTH], \THEYEAR}


% fancy header 

%\usepackage{fancyhdr}
%\pagestyle{fancy}
%\renewcommand{\headrulewidth}{0pt}
%\fancyhead{}
%\fancyfoot{}
%
%
%
%\fancyfoot[C]{\thepage}
%\setlength{\headheight}{13.6pt}
%\setlength{\headwidth}{15.92cm}
%%\setlength{\headheight}{25.3pt}
%\chead{\textsc{For the Greta Good}}
%
%
%\fancypagestyle{plain}{%
%	\fancyhead{}
%	\fancyfoot{}
%	\fancyhf{} % clear all header and footer fields
%	\fancyfoot[C]{\thepage} % except the center
%	%\renewcommand{\headrulewidth}{0pt}
%	%\renewcommand{\footrulewidth}{0pt}
%}


%--------------------------------------------------------------------
%	AUTHOR & TITLE
%--------------------------------------------------------------------
\title{– For the Greta Good –\\The Rise of Environmental Awareness
	through Child-to-Parent Value Transmission\footnote{Thank you for funding???
		Lina Leutner, Franziska Wintersteller provided excellent research assistance. All errors and omissions are our own.}}
\author{Marc Fabel, Helmut Rainer, Maria Waldinger}

\date{Last modified: \today}








%--------------------------------------------------------------------
%	BEGIN DOCUMENT
%--------------------------------------------------------------------




\begin{document}
\setcounter{page}{0}  
% \tableofcontents
\newpage
\setcounter{page}{1}    
\maketitle

%\textbf{\color{red} Preliminary and incomplete draft\newline Please do not cite or circulate without the author's permission}
%\renewcommand{\abstractname}{\vspace{-\baselineskip}} % GET RID OF ABSTRACT TITLE

%  \begin{abstract}\noindent 
%   \footnotesize{\begin{center}\textbf{Abstract}\end{center} Place abstract here}
%    \end{abstract}

\bigskip
\tableofcontents

\newpage


%--------------------------------------------------------------------
% INTRODUCTION
%--------------------------------------------------------------------
\section{Introduction [Helmut \& Maria]}\label{sec_greta_cons:introduction}




literature: 
\begin{itemize}
	\item mobile-phone based tracking data: \cite{dave2020contagion}
\end{itemize}



%--------------------------------------------------------------------
% BACKGROUND
%--------------------------------------------------------------------
\bigskip
\section{Background}\label{sec_greta_cons:background}



\textbf{Timeline (important dates)}
\begin{itemize}
	\item 20.08.2018 First time in front of Swedish Parliament
	\item 27.08.2018 First coverage in German newspaper
	\item 3.12.2018-14.12.2018 Katowice (climate conference )
	\item mid December first climate strikes in Germany
	\item 23.01.2019-25.01.2019 Davos (world economic forum)
	\item 01.03.2019 Greta visits Hamburg for climate strike
	\item 15.03.2019 Worldwide climate strikes, more than 1.4 million people involved
	\item 29.03.2019 Greta visits Berlin for climate strike
	\item 16.04.2019 Strasbourg speech at EU
	\item 24.05.2019 2nd Global Climate Strike (for EU elections)
	\item 21.06.2019 Aachen: Climate Justice w/o borders
	\item 14.08.-28.08.2019 Journey across Atlantic 
	\item 20.09.-27.09.2019 Global Week of Climate Action
	\item 23.09.-29.09.2019 NY - UN climate action summit
	\item 29.11.2019 Fourth Global Climate Strike
	\item 02.12.-13.12.2019 Madrid (UN climate change conference )
	\item 11.12.2019 Greta Thunberg Time Person of the Year	
\end{itemize}
see coverage of Greta Thunberg in print media (daily and weekly outlets) in Figures \ref{fig_greta_cons:genios_greta_per_1000_2019} and \ref{fig_greta_cons:genios_greta_per_1000_events_2019}





% notes 
movement initiated by Greta Thunberg, who initiated them in August 2018
she stands as noone else as 

school climate strike movement (FFF)
`The Greta Effect' (e.g. The Guardian )Influence on the world stage

prizes 
two consecutive nominations for the Nobel Peace Prize (2019 \& 2020)







\newpage




\newgeometry{left=0.5cm,right=0.5cm,top=3cm,bottom=3cm} 
\begin{landscape}
	\vspace*{\fill}
	\begin{figure}[H]\centering\caption{Important events surrounding FFF and perception in (social) media}

		
		
	
	
	
	
	\begin{subfigure}[h]{0.85\linewidth}\centering\caption{Print media}
		\includegraphics[width=\linewidth]{descriptive/greta_cons_background_timeline_elections_events.pdf}
	\end{subfigure}
	
	
	\par\bigskip\smallskip % force a bit of vertical whitespace
		\begin{subfigure}[h]{0.43\linewidth}\centering\caption{Print media}
			\includegraphics[width=\linewidth]{descriptive/greta_cons_fff_print_media_articles_2019.pdf}
		\end{subfigure}
		\begin{subfigure}[h]{0.43\linewidth}\centering\caption{Twitter feed of Greta Thunberg}
			\includegraphics[width=\linewidth]{descriptive/greta_cons_twitter_gt_favorites_retweets_2019.pdf}
		\end{subfigure}
		
		\begin{minipage}{0.86\linewidth}
			\scriptsize{\emph{Notes:} Panel a presents a timeline of relevant events over the year 2019. Above the timeline, the scheme lists elections that fall in the sample period (election for the European Parliament, Brandenburg, Saxony, and Thuringia). Below the timeline, there is a series of important events, either far-reaching strikes or critical milestones for Greta Thunberg. Panel b shows the daily share of articles that cover FFF (per thousand articles). An article is defined to report about FFF if it contains at least the keyword `Fridays for Future' (in various notations) or `climate strike'. The blue line stems from a moving average with a window of three days. Panel c plots the weekly number of favorites and retweets of Greta Thunberg's tweets (in thousand).\newline \emph{Source:} Own representation with data from Genios and Twitter.}
		\end{minipage}
	\end{figure}
	\vspace*{\fill}\clearpage
\end{landscape}
\restoregeometry








Appendix gives further insights of the content of Greta Thunberg's tweets
Figure \ref{fig_greta_cons:twitter_greta_thunberg_top_words_hashtags_mentions}: top words, hashtags, and mentions
Figure \ref{fig_greta_cons:twitter_greta_sentiment_length}: sentiment and length of tweets


Twitter feed of other German FFF icons: Appendix Figure \ref{fig_greta_cons:twitter_favorites_activists}
















%--------------------------------------------------------------------
% DATA & VARIABLES
%--------------------------------------------------------------------
\newpage
\section{Data}\label{sec_greta_cons:data} 

year 2019, 401 districts (\textit{Kreise})



\subsection{Mobile-Phone Based Tracking Data}

The mobile-phone based tracking data is obtained from \textit{Teralytics}, which draws on the universe of customers from the \textit{Telefonica O$_2$}-network (market share of 31\% in 2019).\footnote{source: https://de.statista.com/statistik/daten/studie/3028/umfrage/marktanteile-der-netzbetreiber-am-mobilfunkmarkt-in-deutschland-seit-1998/} Teralytics uses machine learning-based technology to transform mobile signals into number of journeys between two locations. For the year of 2019, the data contains 64.4 billion trips. There is a required minimum time between the movements so that they count as a journey and it depends on the mode of transport.\footnote{The required minimum time is 30 minutes by car, 60 minutes by train, and 120 minutes by plane.} 


The data reports the daily number of trips between origin-destination pairs.\footnote{Due to data privacy regulations, Teralytix is not allowed to report journeys if there are less than five trips, given the origin-destination pair. We set the missing counts to zero.} The spatial level of aggregation mostly coincides with district borders, only for larger metropolitan areas, there are regions with smaller clusters. For this reason, the data set provides us the number of journeys for 513 different regions, which result in more than 260,000 possible origin-destination pairs. If the length of journey exceeds 30 km, the data can distinguish between different modes of transport: car (6.4 percent of all trips), train (0.9 percent), and plane (< 0.03 percent). The vast majority of trips exhibit shorter distances or cannot be attributed to on of the previous transportation modes and fall into the category "not classified" (92.7 percent).






\begin{itemize}
	\item provider is typical carrier for young people - with the help of regional market shares, there is extrapolation to entire population
\end{itemize}














\subsection{Strike Data}

% map: all strikes 2019
\begin{figure}[H]\centering
	\caption{Strikes in 2019}\label{fig_greta_cons:fff_strikes_2019}
	\includegraphics[width=0.8\linewidth]{descriptive/greta_cons_fff_strikes_2019.png}
	\begin{minipage}{0.8\linewidth}
		\scriptsize{\emph{Notes:} }
	\end{minipage}
\end{figure}

variation across months of the year in Figure \ref{fig_greta_cons:fff_strikes_months} in the Appendix


3,943 strikes (1,966 authorities, 1,592 fff website and web-archive, 385 social media- filling up information for largest cities that were missing in the database)

% temporal variation of strike number across sources
\begin{figure}[H]\centering
	\caption{Number of strikes across sources}\label{fig_greta_cons:number_strikes_per_source}
	\includegraphics[width=0.8\linewidth]{descriptive/greta_cons_number_strikes_per_source}
	\begin{minipage}{0.8\linewidth}
		\scriptsize{\emph{Notes:} daily number of strikes}
	\end{minipage}
\end{figure}


\subsection{Weather Data}


\subsection{Holidays}


\subsection{Election Outcomes}


\subsection{Media Outlets}


\subsection{Population Figures}






%--------------------------------------------------------------------
% MEASUREMENT STRIKE PARTICIPATION
%--------------------------------------------------------------------
\newpage
\section{Granular Measurement of Strike Participation}\label{sec_greta_cons:measurement_strike_participation}

\subsection{Empirical Strategy}

\subsection{Results}


% fig: strike participation for selected strikes
\begin{figure}[H]\centering
	\caption{Strike participation for selected strikes}
	\label{fig_greta_cons:strike_participation_hh_ber}
	% Hamburg
	\begin{subfigure}[h]{0.45\linewidth}\centering
		\includegraphics[width=\linewidth]{descriptive/greta_cons_strike_participation_hh_ols.png}
	\end{subfigure}
	%Berlin
	\begin{subfigure}[h]{0.45\linewidth}\centering
		\includegraphics[width=\linewidth]{descriptive/greta_cons_strike_participation_ber_ols.png}
	\end{subfigure}
%	%Aachen
%	\begin{subfigure}[h]{0.4\linewidth}\centering
%		\includegraphics[width=\linewidth]{descriptive/greta_cons_strike_participation_aa_ols.png}
%	\end{subfigure}
	\begin{minipage}{0.9\linewidth}
		\scriptsize{\emph{Notes:} Residualized movements [in thousand]. thin lines: ids of teralytics}
	\end{minipage}
\end{figure}


for Hamburg, alternative participation measures shown in Figure \ref{fig_greta_cons:strike_participation_hh_different_measure}

validation exercise: where do spectators at soccer games come from -> Figure \ref{fig_greta_cons:participation_soccer_games}









%--------------------------------------------------------------------
% ASSOCIATION WITH ELECTION OUTCOMES
%--------------------------------------------------------------------
\clearpage
\section{Strike Participation and Electoral Outcomes [Helmut \& Maria]}\label{sec_greta_cons:strike_participation_elections}
\subsection{Empirical Strategy}


\subsection{Results}

% maps greens, fd_greens, strike_index, correlation of the two
\begin{figure}[H]\centering
	\caption{Spatial correlation of the vote share of the Greens and strike participation}
	\label{fig_greta_cons:spatial_correlation_greens_index}
	% greens
	\begin{subfigure}[h]{0.45\linewidth}\centering
		\includegraphics[width=\linewidth]{descriptive/greta_cons_the_greens_eu_election_2019_ags8.png}
	\end{subfigure}
	% fd_greens
	\begin{subfigure}[h]{0.45\linewidth}\centering
		\includegraphics[width=\linewidth]{descriptive/greta_cons_fd_the_greens_eu_election_2019_ags8.png}
	\end{subfigure}
	\begin{subfigure}[h]{0.45\linewidth}\centering
		\includegraphics[width=\linewidth]{descriptive/greta_cons_particiaption_index_eu_election_2019_ags8.png}
	\end{subfigure}
	\begin{subfigure}[h]{0.45\linewidth}\centering
		\includegraphics[width=\linewidth]{descriptive/bivariate_map_legend}
	\end{subfigure}

	\begin{minipage}{0.9\linewidth}
		\scriptsize{\emph{Notes:} quantiles in Panels A-C}
	\end{minipage}
\end{figure}


mention \cite{cantoni2020persistence}


% Tab - first results - assocation strike part & greens
\begin{table}[H]\centering
	\begin{threeparttable}
		\caption{Strike participation and the greens' vote share}\label{tab_greta_cons:associations_part_greens}
		{\def\sym#1{\ifmmode^{#1}\else\(^{#1}\)\fi} 
			\begin{tabular}{l*{3}{c}}
				\toprule
				&\multicolumn{1}{c}{(1)}&\multicolumn{1}{c}{(2)}&\multicolumn{1}{c}{(3)}\\
				& The Greens & $\Delta$ The Greens & $\Delta$ The Greens \\
				& 2019		 & 2019-2015		& 2019-2015 \\
				\midrule
				Strike Participation    &      0.1327\sym{***}&      0.0570\sym{**}	 	&	0.0626\sym{**}	\\
										&    (0.0314)         &    (0.0251)         	&	(0.0313)		\\
				\\	
				Observations        	&      12,189         &      12,189         	&	452				\\
				$R^2$               	&       0.570         &       0.755         	&	0.823			\\
				State FE				& \checkmark 		  & \checkmark       		& \checkmark 		\\
				Election FE				& \checkmark 		  & \checkmark       		& \checkmark 		\\
				\bottomrule
		\end{tabular}}
		\begin{tablenotes} 
			\item \scriptsize \emph{Notes:}  \newline Significance levels: * p < 0.10, ** p < 0.05, *** p < 0.01.
		\end{tablenotes} 
	\end{threeparttable}
\end{table}



% Tab: inclusion controls
\begin{table}[H]\centering
	\begin{threeparttable}
		\caption{Inclusion of controls}\label{tab_greta_cons:inclusion_controls}
		{\def\sym#1{\ifmmode^{#1}\else\(^{#1}\)\fi} 
			\begin{tabular}{l*{4}{c}}
				\toprule
				&\multicolumn{1}{c}{(1)}&\multicolumn{1}{c}{(2)}&\multicolumn{1}{c}{(3)}&\multicolumn{1}{c}{(4)}\\
				& Baseline & Income & Unemployment &  Demographics \\
				\midrule
				Strike Participation    &      0.0570\sym{**} &      0.0614\sym{***}&      0.0606\sym{**} &      0.0436\sym{*}  \\
										&    (0.0251)         	&    (0.0209)         &    (0.0254)         &    (0.0239)         \\
				\\
				Observations        	&      12,189         &      12,185         &      12,066         &      12,189         \\
				$R^2$              		&       0.755         &       0.800         &       0.755         &       0.778         \\			
				State FE				& \checkmark 		  & \checkmark       & \checkmark 	& \checkmark \\
				Election FE				& \checkmark 		  & \checkmark       & \checkmark   & \checkmark \\
				\bottomrule
		\end{tabular}}
		\begin{tablenotes} 
			\item \scriptsize \emph{Notes:}  \newline Significance levels: * p < 0.10, ** p < 0.05, *** p < 0.01.
		\end{tablenotes} 
	\end{threeparttable}
\end{table}




% Tab: other participation measures
\begin{table}[H]\centering
	\begin{threeparttable}
		\caption{Alternative participation measures}\label{tab_greta_cons:alternative_participation_measures}
		{\def\sym#1{\ifmmode^{#1}\else\(^{#1}\)\fi} 
			\begin{tabular}{l*{3}{c}}
				\toprule
				&\multicolumn{1}{c}{(1)}&\multicolumn{1}{c}{(2)}&\multicolumn{1}{c}{(3)}\\
				& Baseline & \clb{c}{Partial\\interaction} & \clb{c}{Fully\\interacted} \\
				\midrule\\
				
				% OLS
				\multicolumn{4}{l}{\textbf{\textit{Panel A: OLS}}} \\
				Strike participation    &      0.0570\sym{**} &      0.0586\sym{**} &      0.0497\sym{*}  \\  
										&    (0.0251)         &    (0.0260)         &    (0.0290)         \\  
				Observations      		&      12,189         &      12,189         &      12,189         \\  
				$R^2$             		&       0.755         &       0.755         &       0.754         \\ 
				State FE				& \checkmark 		  & \checkmark       & \checkmark  \\
				Election FE				& \checkmark 		  & \checkmark       & \checkmark  \\
				\\ 
				
				% Poisson
				\multicolumn{4}{l}{\textbf{\textit{Panel B: Poisson}}} \\
				Strike participation    &      0.0460\sym{*}  &      0.0482\sym{**} &      0.0475		  \\
										&    (0.0241)         &    (0.0244)         &    (0.0293)         \\
				Observations      		&      12,189         &      12,189         &      12,189         \\
				$R^2$             		&       0.754         &       0.754         &       0.754         \\		
				State FE				& \checkmark 		  & \checkmark       & \checkmark  \\
				Election FE				& \checkmark 		  & \checkmark       & \checkmark  \\
				\bottomrule
		\end{tabular}}
		\begin{tablenotes} 
			\item \scriptsize \emph{Notes:}  \newline Significance levels: * p < 0.10, ** p < 0.05, *** p < 0.01.
		\end{tablenotes} 
	\end{threeparttable}
\end{table}



% figure cutoff distances
\begin{figure}[H]\centering
	\caption{Cutoff distances for origins}\label{fig_greta_cons:origin_cutoff_distances}
	\includegraphics[width=0.9\linewidth]{analysis/greta_cons_cutoff_distance_election_outcomes}
	\begin{minipage}{0.9\linewidth}
		\scriptsize{\emph{Notes:} }
	\end{minipage}
\end{figure}





%--------------------------------------------------------------------
% CONCLUSION
%-------------------------------------------------------------------
\bigskip
\section{Concluding remarks [Helmut \& Maria]}\label{sec_greta_cons:conclusion}




%--------------------------------------------------------------------
% BIBLIOGRAPHY
%--------------------------------------------------------------------
\newpage


\bibliographystyle{ecca_edited}%previous style-chicago
\bibliography{greta_cons_bibliography}








%--------------------------------------------------------------------
% APPENDIX
%--------------------------------------------------------------------
\newpage
\TODO\section{Appendix}
\vspace*{\fill}
{\Huge \begin{center}\textbf{APPENDIX}\end{center}}
\vspace*{\fill}\clearpage


\renewcommand\thefigure{A.\arabic{figure}}
\setcounter{figure}{0} 
\captionsetup[subfigure]{labelformat=parens}




% GRETA_CONS_APPENDIX_FIGURES


%--------------------------------------------
% Greta tweets - top words, hashtags, mentions
\newgeometry{left=0.5cm,right=0.5cm,top=3cm,bottom=3cm} 
\begin{landscape}
	\vspace*{\fill}
	\begin{figure}[H]
		\centering
		\caption{Most widely used words, hashtags, and mentions in Greta Thunberg's tweets}
		\label{fig_greta_cons:twitter_greta_thunberg_top_words_hashtags_mentions}
		
		\begin{subfigure}[h]{0.4\linewidth}\centering\caption{Word cloud}
			\includegraphics[width=\linewidth]{descriptive/greta_cons_twitter_greta_word_cloud.pdf}
		\end{subfigure}
		\begin{subfigure}[h]{0.4\linewidth}\centering\caption{Top words}
			\includegraphics[width=\linewidth]{descriptive/greta_cons_twitter_greta_frequency_common_words.pdf}
		\end{subfigure}
	
		\begin{subfigure}[h]{0.4\linewidth}\centering\caption{Top hashtags}
			\includegraphics[width=\linewidth]{descriptive/greta_cons_twitter_greta_frequency_common_hashtags.pdf}
		\end{subfigure}
		\begin{subfigure}[h]{0.4\linewidth}\centering\caption{Top mentions}
			\includegraphics[width=\linewidth]{descriptive/greta_cons_twitter_greta_frequency_common_mentions.pdf}
		\end{subfigure}
		\scriptsize
		\begin{minipage}{0.95\linewidth}
			\scriptsize{\emph{Notes:} Panel a displays a word cloud of Greta Thunberg's tweets in 2019. The larger a word is depicted, the more frequent it is used by Greta Thunberg in her tweets. The lowercased keywords are from the subset of words after removing urls, hashtags, mentions, digits, punctuation, and stopwords. Panels b, c, and d show the frequency of the most common words, hashtags, and mentions. The people behind the Twitter accounts are: @$\_$NikkiHenderson, @Sailing$\_$LaVaga, @elayna$\_$c (sailors), @KevinClimate (Professor of energy and climate change), @Luisamneubauer (German climate activist), @AnunaDe (Belgian climate activist), @GeorgeMonbiot (British writer and political activist), @jrockstrom (Director of Potsdam Institute), @SKAVLANTVShow (Scandinavian talk show), @MichaelEMann (Professor of Atmospheric Science).\newline\emph{Source:} Own representation with data from Twitter.}
		\end{minipage}
	\end{figure}
	\vspace*{\fill}\clearpage
\end{landscape}
\restoregeometry
%--------------------------------------------
% Greta tweets - sentiment and length of tweets
\vspace*{\fill}
\begin{figure}[H]
	\centering\caption{Sentiment and length of Greta Thunberg's tweets}
	\label{fig_greta_cons:twitter_greta_sentiment_length}
	\includegraphics[width=\linewidth]{descriptive/greta_cons_twitter_activists_scatter_polarity_subjectivity_word_count.pdf}
	\begin{minipage}{0.99\linewidth}
		\scriptsize{\emph{Notes:} The two top panels show the conditional distribution of the tweets' sentiment. Greta Thunberg's tweets tend to be more positively formulated (polarity values larger than zero), with the right balance between objective and subjective language. Her tweets contain on average 135 characters and a bit more than 20 words.\newline\emph{Source:} Own representation with data from Twitter.}
	\end{minipage}
\end{figure}
\vspace*{\fill}\clearpage
%--------------------------------------------
% twitter likes of German FFF ICONS
\vspace*{\fill}
\begin{figure}[H]
	\centering\caption{The twitter feed of influential German FFF activists}\label{fig_greta_cons:twitter_favorites_activists}
	\includegraphics[width=\linewidth]{descriptive/greta_cons_twitter_favorites_spaghetti_wo_greta_2019.pdf}
	\begin{minipage}{0.99\linewidth}
		\scriptsize{\emph{Notes:} The figure plots the weekly number of likes [in thousand] of influential German climate activists and groups over the year 2019.\newline\emph{Source:} Own representation with data from Twitter.}
	\end{minipage}
\end{figure}
\vspace*{\fill}\clearpage
%--------------------------------------------
% data - map: strikes 2019 across months
\vspace*{\fill}
\begin{figure}[H]\centering
	\caption{Strikes in 2019, variation across months}\label{fig_greta_cons:fff_strikes_months}
	\includegraphics[width=0.99\linewidth]{descriptive/greta_cons_fff_strikes_all_months.png}
	\begin{minipage}{0.99\linewidth}
		\scriptsize{\emph{Notes:} The maps show the climate strikes (red dots) in our data base over the year 2019. The bold white lines indicate state boundaries and the thin white lines represent county boundaries.}
	\end{minipage}
\end{figure}
\vspace*{\fill}\clearpage
%--------------------------------------------
% analysis strike particpation robustness: alternative participation measures
\newgeometry{left=0.5cm,right=0.5cm,top=3cm,bottom=3cm} 
\begin{landscape}
	\vspace*{\fill}
	\begin{figure}[H]\centering
		\caption{Alternative participation measures for the climate strike in Berlin}
		\label{fig_greta_cons:strike_participation_ber_different_measure}
		% Hamburg
		\begin{subfigure}[h]{0.22\linewidth}\centering
			\includegraphics[width=\linewidth]{descriptive/maps_resid_trips/greta_cons_strike_participation_ber_spec_res_ols}
		\end{subfigure}
		\begin{subfigure}[h]{0.22\linewidth}\centering
			\includegraphics[width=\linewidth]{descriptive/maps_resid_trips/greta_cons_strike_participation_ber_spec_res_ols_int_small}
		\end{subfigure}
		\begin{subfigure}[h]{0.22\linewidth}\centering
			\includegraphics[width=\linewidth]{descriptive/maps_resid_trips/greta_cons_strike_participation_ber_spec_res_ols_int_only}
		\end{subfigure}
		
		\begin{subfigure}[h]{0.22\linewidth}\centering
			\includegraphics[width=\linewidth]{descriptive/maps_resid_trips/greta_cons_strike_participation_ber_spec_res_p}
		\end{subfigure}
		\begin{subfigure}[h]{0.22\linewidth}\centering
			\includegraphics[width=\linewidth]{descriptive/maps_resid_trips/greta_cons_strike_participation_ber_spec_res_p_int_small}
		\end{subfigure}
		\begin{subfigure}[h]{0.22\linewidth}\centering
			\includegraphics[width=\linewidth]{descriptive/maps_resid_trips/greta_cons_strike_participation_ber_spec_res_p_int_only}
		\end{subfigure}
		\begin{minipage}{0.9\linewidth}
			\scriptsize{\emph{Notes:} The maps show residualized movements (in thousand) for the climate strike in Berlin (March, 29). Column 1 uses the baseline specification in Equation \ref{eq_greta_cons:res_journeys}, column 2 further adds $\vartheta_{ij}$$\times\eta_w$ and  $\vartheta_{ij}$$\times\psi_m$, and column 3 presents a fully interacted version (plus also including $\vartheta_{ij}$$\times\varphi_d$). The top row uses OLS to estimate Equation \ref{eq_greta_cons:res_journeys}, while the bottom row shows results when using Poisson. A darker shade of green indicates that more people were coming to the climate strike conditional on the controls discussed in the text. The color scale classification is obtained by using the Fisher-Jenks natural breaks algorithm. See Figure \ref{fig_greta_cons:strike_participation_hh_ber} for additional details.}
		\end{minipage}
	\end{figure}
	\vspace*{\fill}\clearpage
\end{landscape}
\restoregeometry
%--------------------------------------------
% analysis strike particpation validation: soccer games

\begin{figure}[H]\centering
	\caption{Validation: Participation soccer games}
	\label{fig_greta_cons:participation_soccer_games}
	\begin{subfigure}[h]{0.45\linewidth}\centering
		\includegraphics[width=\linewidth]{descriptive/greta_cons_soccer_muc_jan25.png}
	\end{subfigure}
	\begin{subfigure}[h]{0.45\linewidth}\centering
		\includegraphics[width=\linewidth]{descriptive/greta_cons_soccer_dtm_feb01.png}
	\end{subfigure}
	\begin{subfigure}[h]{0.45\linewidth}\centering
		\includegraphics[width=\linewidth]{descriptive/greta_cons_soccer_fri_feb22.png}
	\end{subfigure}
	
	\begin{minipage}{0.9\linewidth}
		\scriptsize{\emph{Notes:} The maps show residualized movements (in thousand) for soccer matches in Munich, Dortmund, and Freiburg. A darker shade of green indicates that more people were coming to the climate strike conditional on the controls discussed in the text. The color scale classification is obtained by using the Fisher-Jenks natural breaks algorithm. The red dots mark the games' location, gray areas indicate missing data (censored), bold gray lines show state boundaries, and thin gray lines represent the regions defined by \textit{Teralytics}.}
	\end{minipage}
\end{figure}






\end{document}